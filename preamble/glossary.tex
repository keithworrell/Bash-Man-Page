% Glossaries Section
\makeglossaries
\newglossaryentry{bash}{
    name=Bash,
    text=bash,
    description=Bash is an sh-compatible command language interpreter that executes commands read from the standard input or from a file. Bash also incorporates useful features from the Korn and C shells (ksh and csh).\newline\newline Bash is intended to be a conformant implementation of the Shell and Utilities portion of the IEEE POSIX specification (IEEE Standard 1003.1). Bash can be configured to be POSIX-conformant by default.
}

\newglossaryentry{sh}{
    name=sh,
    text=sh,
    description={Bash is an sh-compatible command language interpreter that executes commands read from the standard input or from a file. Bash also incorporates useful features from the Korn and C shells (ksh and csh).\newline\newline Bash is intended to be a conformant implementation of the Shell and Utilities portion of the IEEE POSIX specification (IEEE Standard 1003.1). Bash can be configured to be POSIX-conformant by default.}
}

\newglossaryentry{ksh}{
    name=ksh,
    text=ksh,
    description={ksh is a command interpreter that is intended for both interactive and shell script use. Its command language is a superset of the sh(1) shell language.}
}

\newglossaryentry{csh}{
    name=csh,
    description={tcsh is an enhanced but completely compatible version of the Berkeley UNIX C shell, csh(1). It is a command language interpreter usable both as an interactive login shell and a shell script command processor. It includes a command-line editor (see The command-line editor), programmable word completion (see Completion and listing), spelling correction (see Spelling correction), a history mechanism (see History substitution), job control (see Jobs) and a C-like syntax. The NEW FEATURES section describes major enhancements of tcsh over csh(1). Throughout this manual, features of tcsh not found in most csh(1) implementations (specifically, the 4.4BSD csh) are labeled with '(+)', and features which are present in csh(1) but not usually documented are labeled with '(u)'.}
}

\newglossaryentry{set}{
    name=set,
    text=set,
    description={}
}

\newglossaryentry{PATH}{
    name=PATH,
    text=PATH,
    description={}
}

\newglossaryentry{isatty}{
    name=isatty,
    text={isatty(3)},
    description={The isatty() function tests whether fd is an open file descriptor referring to a terminal.}
}

\newglossaryentry{PS1}{
    name=PS1,
    text=PS1,
    description={}
}


\renewcommand{\glstextformat}[1]{\textbf{#1}}

% End Glossary Section