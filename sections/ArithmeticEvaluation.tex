\section{Arithmetic Evaluation}\label{sec:arithmeticevaluation}
The shell allows arithmetic expressions to be evaluated, under certain circumstances (see the let and declare builtin commands and Arithmetic Expansion). Evaluation is done in fixed-width integers with no check for overflow, though division by 0 is trapped and flagged as an error. The operators and their precedence, associativity, and values are the same as in the C language. The following list of operators is grouped into levels of equal-precedence operators. The levels are listed in order of decreasing precedence.

\begin{longtable}\defaultlongtable
id++ id-- &
variable post-increment and post-decrement \\

++id --id &
variable pre-increment and pre-decrement \\

- + &
unary minus and plus \\

! ~ &
logical and bitwise negation \\

** &
exponentiation \\

* / \% &
multiplication, division, remainder \\

+ - &
addition, subtraction \\

<< >> &
left and right bitwise shifts \\

<= >= < > &
comparison \\

== \!= &
equality and inequality \\

\& &
bitwise AND \\

\^{} &
bitwise exclusive OR \\

| &
bitwise OR \\

\&\& &
logical AND \\

|| &
logical OR \\

expr?expr:expr &
conditional operator \\

= *= /= \%= += -= <<= >>= \&= \^{}= |= &
assignment \\

expr1 , expr2 &
comma \\
\end{longtable}

Shell variables are allowed as operands; parameter expansion is performed before the expression is evaluated. Within an expression, shell variables may also be referenced by name without using the parameter expansion syntax. A shell variable that is null or unset evaluates to 0 when referenced by name without using the parameter expansion syntax. The value of a variable is evaluated as an arithmetic expression when it is referenced, or when a variable which has been given the integer attribute using declare -i is assigned a value. A null value evaluates to 0. A shell variable need not have its integer attribute turned on to be used in an expression.
Constants with a leading 0 are interpreted as octal numbers. A leading 0x or 0X denotes hexadecimal. Otherwise, numbers take the form [base\#]n, where base is a decimal number between 2 and 64 representing the arithmetic base, and n is a number in that base. If base\# is omitted, then base 10 is used. The digits greater than 9 are represented by the lowercase letters, the uppercase letters, @, and \_, in that order. If base is less than or equal to 36, lowercase and uppercase letters may be used interchangeably to represent numbers between 10 and 35.

Operators are evaluated in order of precedence. Sub-expressions in parentheses are evaluated first and may override the precedence rules above.

Conditional Expressions
Conditional expressions are used by the [[ compound command and the test and [ builtin commands to test file attributes and perform string and arithmetic comparisons. Expressions are formed from the following unary or binary primaries. If any file argument to one of the primaries is of the form /dev/fd/n, then file descriptor n is checked. If the file argument to one of the primaries is one of /dev/stdin, /dev/stdout, or /dev/stderr, file descriptor 0, 1, or 2, respectively, is checked.

Unless otherwise specified, primaries that operate on files follow symbolic links and operate on the target of the link, rather than the link itself.

When used with [[, The < and > operators sort lexicographically using the current locale.

-a file
True if file exists.
-b file
True if file exists and is a block special file.
-c file
True if file exists and is a character special file.
-d file
True if file exists and is a directory.
-e file
True if file exists.
-f file
True if file exists and is a regular file.
-g file
True if file exists and is set-group-id.
-h file
True if file exists and is a symbolic link.
-k file
True if file exists and its ''sticky'' bit is set.
-p file
True if file exists and is a named pipe (FIFO).
-r file
True if file exists and is readable.
-s file
True if file exists and has a size greater than zero.
-t fd
True if file descriptor fd is open and refers to a terminal.

-u file
True if file exists and its set-user-id bit is set.
-w file
True if file exists and is writable.
-x file
True if file exists and is executable.
-O file
True if file exists and is owned by the effective user id.
-G file
True if file exists and is owned by the effective group id.
-L file
True if file exists and is a symbolic link.
-S file
True if file exists and is a socket.
-N file
True if file exists and has been modified since it was last read.
file1 -nt file2
True if file1 is newer (according to modification date) than file2, or if file1 exists and file2 does not.
file1 -ot file2
True if file1 is older than file2, or if file2 exists and file1 does not.
file1 -ef file2
True if file1 and file2 refer to the same device and inode numbers.
-o optname
True if shell option optname is enabled. See the list of options under the description of the -o option to the \hyperref[sbc:set]{set} builtin below.
-z string
True if the length of string is zero.
string
-n string
True if the length of string is non-zero.
string1 == string2
string1 = string2
True if the strings are equal. = should be used with the test command for POSIX conformance.
string1 != string2
True if the strings are not equal.
string1 < string2
True if string1 sorts before string2 lexicographically.
string1 > string2
True if string1 sorts after string2 lexicographically.
arg1 OP arg2
OP is one of -eq, -ne, -lt, -le, -gt, or -ge. These arithmetic binary operators return true if arg1 is equal to, not equal to, less than, less than or equal to, greater than, or greater than or equal to arg2, respectively. Arg1 and arg2 may be positive or negative integers.
Simple Command Expansion
When a simple command is executed, the shell performs the following expansions, assignments, and redirections, from left to right.

\begin{enumerate}
\item  The words that the parser has marked as variable assignments (those preceding the command name) and redirections are saved for later processing.

\item  The words that are not variable assignments or redirections are expanded. If any words remain after expansion, the first word is taken to be the name of the command and the remaining words are the arguments.

\item  Redirections are performed as described above under REDIRECTION.

\item  The text after the = in each variable assignment undergoes tilde expansion, parameter expansion, command substitution, arithmetic expansion, and quote removal before being assigned to the variable.
\end{enumerate}

If no command name results, the variable assignments affect the current shell environment. Otherwise, the variables are added to the environment of the executed command and do not affect the current shell environment. If any of the assignments attempts to assign a value to a readonly variable, an error occurs, and the command exits with a non-zero status.
If no command name results, redirections are performed, but do not affect the current shell environment. A redirection error causes the command to exit with a non-zero status.

If there is a command name left after expansion, execution proceeds as described below. Otherwise, the command exits. If one of the expansions contained a command substitution, the exit status of the command is the exit status of the last command substitution performed. If there were no command substitutions, the command exits with a status of zero.