\section{Readline}\label{sec:readline}
This is the library that handles reading input when using an interactive shell, unless the --noediting option is given at shell invocation. Line editing is also used when using the -e option to the read builtin. By default, the line editing commands are similar to those of emacs. A vi-style line editing interface is also available. Line editing can be enabled at any time using the -o emacs or -o vi options to the set builtin \hyperref[sec:shellbuiltincommands]{see SHELL BUILTIN COMMANDS below)}. To turn off line editing after the shell is running, use the +o emacs or +o vi options to the set builtin.

\subsection{Readline Notation}\label{sec:readlinenotation}

In this section, the emacs-style notation is used to denote keystrokes. Control keys are denoted by C-key, e.g., C-n means Control-N. Similarly, meta keys are denoted by M-key, so M-x means Meta-X. (On keyboards without a meta key, M-x means ESC x, i.e., press the Escape key then the x key. This makes ESC the meta prefix. The combination M-C-x means ESC-Control-x, or press the Escape key then hold the Control key while pressing the x key.)
Readline commands may be given numeric arguments, which normally act as a repeat count. Sometimes, however, it is the sign of the argument that is significant. Passing a negative argument to a command that acts in the forward direction (e.g., kill-line) causes that command to act in a backward direction. Commands whose behavior with arguments deviates from this are noted below.

When a command is described as killing text, the text deleted is saved for possible future retrieval (yanking). The killed text is saved in a kill ring. Consecutive kills cause the text to be accumulated into one unit, which can be yanked all at once. Commands which do not kill text separate the chunks of text on the kill ring.

\subsection{Readline Initialization}\label{sec:readlineinitialization}

Readline is customized by putting commands in an initialization file (the inputrc file). The name of this file is taken from the value of the INPUTRC variable. If that variable is unset, the default is ~/.inputrc. When a program which uses the readline library starts up, the initialization file is read, and the key bindings and variables are set. There are only a few basic constructs allowed in the readline initialization file. Blank lines are ignored. Lines beginning with a \# are comments. Lines beginning with a \$ indicate conditional constructs. Other lines denote key bindings and variable settings.
The default key-bindings may be changed with an inputrc file. Other programs that use this library may add their own commands and bindings.

For example, placing

\url{M-Control-u: universal-argument}
or
\url{C-Meta-u: universal-argument}
into the inputrc would make M-C-u execute the readline command universal-argument.
The following symbolic character names are recognized: RUBOUT, DEL, ESC, LFD, NEWLINE, RET, RETURN, SPC, SPACE, and TAB.

In addition to command names, readline allows keys to be bound to a string that is inserted when the key is pressed (a macro).

\subsection{Readline Key Bindings}\label{sec:readlinekeybindings}

The syntax for controlling key bindings in the inputrc file is simple. All that is required is the name of the command or the text of a macro and a key sequence to which it should be bound. The name may be specified in one of two ways: as a symbolic key name, possibly with Meta- or Control- prefixes, or as a key sequence.
When using the form keyname:function-name or macro, keyname is the name of a key spelled out in English. For example:

\begin{lstlisting}
Control-u: universal-argument
Meta-Rubout: backward-kill-word
Control-o: "> output"
\end{lstlisting}

In the above example, C-u is bound to the function universal-argument, M-DEL is bound to the function backward-kill-word, and C-o is bound to run the macro expressed on the right hand side (that is, to insert the text ''> output'' into the line).
In the second form, "keyseq":function-name or macro, keyseq differs from keyname above in that strings denoting an entire key sequence may be specified by placing the sequence within double quotes. Some GNU Emacs style key escapes can be used, as in the following example, but the symbolic character names are not recognized.

\begin{lstlisting}
"\C-u": universal-argument
"\C-x\C-r": re-read-init-file
"\e[11~": "Function Key 1"
\end{lstlisting}
In this example, C-u is again bound to the function universal-argument. C-x C-r is bound to the function re-read-init-file, and ESC [ 1 1 ~ is bound to insert the text ''Function Key 1''.
The full set of GNU Emacs style escape sequences is
\begin{longtable}{p{0.3\textwidth}p{0.3\textwidth}}
\textbackslash C- &
control prefix \\

\textbackslash M- &
meta prefix \\

\textbackslash e &
an escape character \\

\textbackslash \textbackslash &
backslash \\

\textbackslash " &
literal " \\

\textbackslash ' &
literal ' \\
\end{longtable}
In addition to the GNU Emacs style escape sequences, a second set of backslash escapes is available:

\begin{longtable}{p{0.3\textwidth}p{0.3\textwidth}}
\textbackslash a &
alert (bell) \\

\textbackslash b &
backspace \\

\textbackslash d &
delete \\

\textbackslash f &
form feed \\

\textbackslash n &
newline \\

\textbackslash r &
carriage return \\

\textbackslash t &
horizontal tab \\

\textbackslash v &
vertical tab \\

\textbackslash nnn &
the eight-bit character whose value is the octal value nnn (one to three digits) \\

\textbackslash xHH &
the eight-bit character whose value is the hexadecimal value HH (one or two hex digits) \\
\end{longtable}

When entering the text of a macro, single or double quotes must be used to indicate a macro definition. Unquoted text is assumed to be a function name. In the macro body, the backslash escapes described above are expanded. Backslash will quote any other character in the macro text, including " and '.
Bash allows the current readline key bindings to be displayed or modified with the bind builtin command. The editing mode may be switched during interactive use by using the -o option to the set builtin command \hyperref[sec:shellbuiltincommands]{see SHELL BUILTIN COMMANDS below)}.

\subsection{Readline Variables}\label{sec:readlinevariables}

Readline has variables that can be used to further customize its behavior. A variable may be set in the inputrc file with a statement of the form
set variable-name value
Except where noted, readline variables can take the values On or Off (without regard to case). Unrecognized variable names are ignored. When a variable value is read, empty or null values, "on" (case-insensitive), and "1" are equivalent to On. All other values are equivalent to Off. The variables and their default values are:
bell-style (audible)
Controls what happens when readline wants to ring the terminal bell. If set to none, readline never rings the bell. If set to visible, readline uses a visible bell if one is available. If set to audible, readline attempts to ring the terminal's bell.
bind-tty-special-chars (On)
If set to On, readline attempts to bind the control characters treated specially by the kernel's terminal driver to their readline equivalents.
comment-begin (''\#'')
The string that is inserted when the readline insert-comment command is executed. This command is bound to M-\# in emacs mode and to \# in vi command mode.
completion-ignore-case (Off)
If set to On, readline performs filename matching and completion in a case-insensitive fashion.
completion-prefix-display-length (0)
The length in characters of the common prefix of a list of possible completions that is displayed without modification. When set to a value greater than zero, common prefixes longer than this value are replaced with an ellipsis when displaying possible completions.
completion-query-items (100)
This determines when the user is queried about viewing the number of possible completions generated by the possible-completions command. It may be set to any integer value greater than or equal to zero. If the number of possible completions is greater than or equal to the value of this variable, the user is asked whether or not he wishes to view them; otherwise they are simply listed on the terminal.
convert-meta (On)
If set to On, readline will convert characters with the eighth bit set to an ASCII key sequence by stripping the eighth bit and prefixing an escape character (in effect, using escape as the meta prefix).
disable-completion (Off)
If set to On, readline will inhibit word completion. Completion characters will be inserted into the line as if they had been mapped to self-insert.
editing-mode (emacs)
Controls whether readline begins with a set of key bindings similar to emacs or vi. editing-mode can be set to either emacs or vi.
echo-control-characters (On)
When set to On, on operating systems that indicate they support it, readline echoes a character corresponding to a signal generated from the keyboard.
enable-keypad (Off)
When set to On, readline will try to enable the application keypad when it is called. Some systems need this to enable the arrow keys.
enable-meta-key (On)
When set to On, readline will try to enable any meta modifier key the terminal claims to support when it is called. On many terminals, the meta key is used to send eight-bit characters.
expand-tilde (Off)
If set to on, tilde expansion is performed when readline attempts word completion.
history-preserve-point (Off)
If set to on, the history code attempts to place point at the same location on each history line retrieved with previous-history or next-history.
history-size (0)
Set the maximum number of history entries saved in the history list. If set to zero, the number of entries in the history list is not limited.
horizontal-scroll-mode (Off)
When set to On, makes readline use a single line for display, scrolling the input horizontally on a single screen line when it becomes longer than the screen width rather than wrapping to a new line.
input-meta (Off)
If set to On, readline will enable eight-bit input (that is, it will not strip the high bit from the characters it reads), regardless of what the terminal claims it can support. The name meta-flag is a synonym for this variable.
isearch-terminators (''C-[C-J'')
The string of characters that should terminate an incremental search without subsequently executing the character as a command. If this variable has not been given a value, the characters ESC and C-J will terminate an incremental search.
keymap (emacs)
Set the current readline keymap. The set of valid keymap names is emacs, emacs-standard, emacs-meta, emacs-ctlx, vi, vi-command, and vi-insert. vi is equivalent to vi-command; emacs is equivalent to emacs-standard. The default value is emacs; the value of editing-mode also affects the default keymap.
mark-directories (On)
If set to On, completed directory names have a slash appended.
mark-modified-lines (Off)
If set to On, history lines that have been modified are displayed with a preceding asterisk (*).
mark-symlinked-directories (Off)
If set to On, completed names which are symbolic links to directories have a slash appended (subject to the value of mark-directories).
match-hidden-files (On)
This variable, when set to On, causes readline to match files whose names begin with a '.' (hidden files) when performing filename completion, unless the leading '.' is supplied by the user in the filename to be completed.
output-meta (Off)
If set to On, readline will display characters with the eighth bit set directly rather than as a meta-prefixed escape sequence.
page-completions (On)
If set to On, readline uses an internal more-like pager to display a screenful of possible completions at a time.
print-completions-horizontally (Off)
If set to On, readline will display completions with matches sorted horizontally in alphabetical order, rather than down the screen.
revert-all-at-newline (Off)
If set to on, readline will undo all changes to history lines before returning when accept-line is executed. By default, history lines may be modified and retain individual undo lists across calls to readline.
show-all-if-ambiguous (Off)
This alters the default behavior of the completion functions. If set to on, words which have more than one possible completion cause the matches to be listed immediately instead of ringing the bell.
show-all-if-unmodified (Off)
This alters the default behavior of the completion functions in a fashion similar to show-all-if-ambiguous. If set to on, words which have more than one possible completion without any possible partial completion (the possible completions don't share a common prefix) cause the matches to be listed immediately instead of ringing the bell.
skip-completed-text (Off)
If set to On, this alters the default completion behavior when inserting a single match into the line. It's only active when performing completion in the middle of a word. If enabled, readline does not insert characters from the completion that match characters after point in the word being completed, so portions of the word following the cursor are not duplicated.
visible-stats (Off)
If set to On, a character denoting a file's type as reported by stat(2) is appended to the filename when listing possible completions.
Readline Conditional Constructs

Readline implements a facility similar in spirit to the conditional compilation features of the C preprocessor which allows key bindings and variable settings to be performed as the result of tests. There are four parser directives used.
\$if
The \$if construct allows bindings to be made based on the editing mode, the terminal being used, or the application using readline. The text of the test extends to the end of the line; no characters are required to isolate it.

mode
The mode= form of the \$if directive is used to test whether readline is in emacs or vi mode. This may be used in conjunction with the set keymap command, for instance, to set bindings in the emacs-standard and emacs-ctlx keymaps only if readline is starting out in emacs mode.

term

The term= form may be used to include terminal-specific key bindings, perhaps to bind the key sequences output by the terminal's function keys. The word on the right side of the = is tested against the both full name of the terminal and the portion of the terminal name before the first -. This allows sun to match both sun and sun-cmd, for instance.

application
The application construct is used to include application-specific settings. Each program using the readline library sets the application name, and an initialization file can test for a particular value. This could be used to bind key sequences to functions useful for a specific program. For instance, the following command adds a key sequence that quotes the current or previous word in Bash:

\begin{lstlisting}[language=bash]
    $if Bash
    # Quote the current or previous word
    "\C-xq": "\eb"\ef""
    $endif
\end{lstlisting} %syntax highlighting above is incorrect.


\$endif
This command, as seen in the previous example, terminates an \$if command.

\$else

Commands in this branch of the \$if directive are executed if the test fails.

\$include
This directive takes a single filename as an argument and reads commands and bindings from that file. For example, the following directive would read /etc/inputrc:
\$include  /etc/inputrc
Searching

Readline provides commands for searching through the command history \hyperref[sec:history]{see HISTORY below)} for lines containing a specified string. There are two search modes: incremental and non-incremental.
Incremental searches begin before the user has finished typing the search string. As each character of the search string is typed, readline displays the next entry from the history matching the string typed so far. An incremental search requires only as many characters as needed to find the desired history entry. The characters present in the value of the isearch-terminators variable are used to terminate an incremental search. If that variable has not been assigned a value the Escape and Control-J characters will terminate an incremental search. Control-G will abort an incremental search and restore the original line. When the search is terminated, the history entry containing the search string becomes the current line.

To find other matching entries in the history list, type Control-S or Control-R as appropriate. This will search backward or forward in the history for the next entry matching the search string typed so far. Any other key sequence bound to a readline command will terminate the search and execute that command. For instance, a newline will terminate the search and accept the line, thereby executing the command from the history list.

Readline remembers the last incremental search string. If two Control-Rs are typed without any intervening characters defining a new search string, any remembered search string is used.

Non-incremental searches read the entire search string before starting to search for matching history lines. The search string may be typed by the user or be part of the contents of the current line.

Readline Command Names

The following is a list of the names of the commands and the default key sequences to which they are bound. Command names without an accompanying key sequence are unbound by default. In the following descriptions, point refers to the current cursor position, and mark refers to a cursor position saved by the set-mark command. The text between the point and mark is referred to as the region.
Commands for Moving

beginning-of-line (C-a)
Move to the start of the current line.
end-of-line (C-e)
Move to the end of the line.
forward-char (C-f)
Move forward a character.
backward-char (C-b)
Move back a character.
forward-word (M-f)
Move forward to the end of the next word. Words are composed of alphanumeric characters (letters and digits).
backward-word (M-b)
Move back to the start of the current or previous word. Words are composed of alphanumeric characters (letters and digits).
shell-forward-word
Move forward to the end of the next word. Words are delimited by non-quoted shell metacharacters.
shell-backward-word
Move back to the start of the current or previous word. Words are delimited by non-quoted shell metacharacters.
clear-screen (C-l)
Clear the screen leaving the current line at the top of the screen. With an argument, refresh the current line without clearing the screen.
redraw-current-line
Refresh the current line.
Commands for Manipulating the History

accept-line (Newline, Return)
Accept the line regardless of where the cursor is. If this line is non-empty, add it to the history list according to the state of the HISTCONTROL variable. If the line is a modified history line, then restore the history line to its original state.
previous-history (C-p)
Fetch the previous command from the history list, moving back in the list.
next-history (C-n)
Fetch the next command from the history list, moving forward in the list.
beginning-of-history (M-<)
Move to the first line in the history.
end-of-history (M->)
Move to the end of the input history, i.e., the line currently being entered.
reverse-search-history (C-r)
Search backward starting at the current line and moving 'up' through the history as necessary. This is an incremental search.
forward-search-history (C-s)
Search forward starting at the current line and moving 'down' through the history as necessary. This is an incremental search.
non-incremental-reverse-search-history (M-p)
Search backward through the history starting at the current line using a non-incremental search for a string supplied by the user.
non-incremental-forward-search-history (M-n)
Search forward through the history using a non-incremental search for a string supplied by the user.
history-search-forward
Search forward through the history for the string of characters between the start of the current line and the point. This is a non-incremental search.
history-search-backward
Search backward through the history for the string of characters between the start of the current line and the point. This is a non-incremental search.
yank-nth-arg (M-C-y)
Insert the first argument to the previous command (usually the second word on the previous line) at point. With an argument n, insert the nth word from the previous command (the words in the previous command begin with word 0). A negative argument inserts the nth word from the end of the previous command. Once the argument n is computed, the argument is extracted as if the "!n" history expansion had been specified.
yank-last-arg (M-., M-\_)
Insert the last argument to the previous command (the last word of the previous history entry). With an argument, behave exactly like yank-nth-arg. Successive calls to yank-last-arg move back through the history list, inserting the last argument of each line in turn. The history expansion facilities are used to extract the last argument, as if the "!\$" history expansion had been specified.
shell-expand-line (M-C-e)
Expand the line as the shell does. This performs alias and history expansion as well as all of the shell word expansions. See HISTORY EXPANSION below for a description of history expansion.
history-expand-line (M-\^{})
Perform history expansion on the current line. See HISTORY EXPANSION below for a description of history expansion.
magic-space
Perform history expansion on the current line and insert a space. See HISTORY EXPANSION below for a description of history expansion.
alias-expand-line
Perform alias expansion on the current line. See ALIASES above for a description of alias expansion.
history-and-alias-expand-line
Perform history and alias expansion on the current line.
insert-last-argument (M-., M-\_)
A synonym for yank-last-arg.
operate-and-get-next (C-o)
Accept the current line for execution and fetch the next line relative to the current line from the history for editing. Any argument is ignored.
edit-and-execute-command (C-xC-e)
Invoke an editor on the current command line, and execute the result as shell commands. Bash attempts to invoke \$VISUAL, \$EDITOR, and emacs as the editor, in that order.
Commands for Changing Text

delete-char (C-d)
Delete the character at point. If point is at the beginning of the line, there are no characters in the line, and the last character typed was not bound to delete-char, then return EOF.
backward-delete-char (Rubout)
Delete the character behind the cursor. When given a numeric argument, save the deleted text on the kill ring.
forward-backward-delete-char
Delete the character under the cursor, unless the cursor is at the end of the line, in which case the character behind the cursor is deleted.
quoted-insert (C-q, C-v)
Add the next character typed to the line verbatim. This is how to insert characters like C-q, for example.
tab-insert (C-v TAB)
Insert a tab character.
self-insert (a, b, A, 1, !, ...)
Insert the character typed.
transpose-chars (C-t)
Drag the character before point forward over the character at point, moving point forward as well. If point is at the end of the line, then this transposes the two characters before point. Negative arguments have no effect.
transpose-words (M-t)
Drag the word before point past the word after point, moving point over that word as well. If point is at the end of the line, this transposes the last two words on the line.
upcase-word (M-u)
Uppercase the current (or following) word. With a negative argument, uppercase the previous word, but do not move point.
downcase-word (M-l)
Lowercase the current (or following) word. With a negative argument, lowercase the previous word, but do not move point.
capitalize-word (M-c)
Capitalize the current (or following) word. With a negative argument, capitalize the previous word, but do not move point.
overwrite-mode
Toggle overwrite mode. With an explicit positive numeric argument, switches to overwrite mode. With an explicit non-positive numeric argument, switches to insert mode. This command affects only emacs mode; vi mode does overwrite differently. Each call to readline() starts in insert mode. In overwrite mode, characters bound to self-insert replace the text at point rather than pushing the text to the right. Characters bound to backward-delete-char replace the character before point with a space. By default, this command is unbound.
Killing and Yanking

kill-line (C-k)
Kill the text from point to the end of the line.
backward-kill-line (C-x Rubout)
Kill backward to the beginning of the line.
unix-line-discard (C-u)
Kill backward from point to the beginning of the line. The killed text is saved on the kill-ring.
kill-whole-line
Kill all characters on the current line, no matter where point is.
kill-word (M-d)
Kill from point to the end of the current word, or if between words, to the end of the next word. Word boundaries are the same as those used by forward-word.
backward-kill-word (M-Rubout)
Kill the word behind point. Word boundaries are the same as those used by backward-word.
shell-kill-word (M-d)
Kill from point to the end of the current word, or if between words, to the end of the next word. Word boundaries are the same as those used by shell-forward-word.
shell-backward-kill-word (M-Rubout)
Kill the word behind point. Word boundaries are the same as those used by shell-backward-word.
unix-word-rubout (C-w)
Kill the word behind point, using white space as a word boundary. The killed text is saved on the kill-ring.
unix-filename-rubout
Kill the word behind point, using white space and the slash character as the word boundaries. The killed text is saved on the kill-ring.
delete-horizontal-space (M-\textbackslash)
Delete all spaces and tabs around point.
kill-region
Kill the text in the current region.
copy-region-as-kill
Copy the text in the region to the kill buffer.
copy-backward-word
Copy the word before point to the kill buffer. The word boundaries are the same as backward-word.
copy-forward-word
Copy the word following point to the kill buffer. The word boundaries are the same as forward-word.
yank (C-y)
Yank the top of the kill ring into the buffer at point.
yank-pop (M-y)
Rotate the kill ring, and yank the new top. Only works following yank or yank-pop.
Numeric Arguments

digit-argument (M-0, M-1, ..., M--)
Add this digit to the argument already accumulating, or start a new argument. M-- starts a negative argument.
universal-argument
This is another way to specify an argument. If this command is followed by one or more digits, optionally with a leading minus sign, those digits define the argument. If the command is followed by digits, executing universal-argument again ends the numeric argument, but is otherwise ignored. As a special case, if this command is immediately followed by a character that is neither a digit or minus sign, the argument count for the next command is multiplied by four. The argument count is initially one, so executing this function the first time makes the argument count four, a second time makes the argument count sixteen, and so on.
Completing

complete (TAB)
Attempt to perform completion on the text before point. Bash attempts completion treating the text as a variable (if the text begins with \$), username (if the text begins with ~), hostname (if the text begins with @), or command (including aliases and functions) in turn. If none of these produces a match, filename completion is attempted.
possible-completions (M-?)
List the possible completions of the text before point.
insert-completions (M-*)
Insert all completions of the text before point that would have been generated by possible-completions.
menu-complete
Similar to complete, but replaces the word to be completed with a single match from the list of possible completions. Repeated execution of menu-complete steps through the list of possible completions, inserting each match in turn. At the end of the list of completions, the bell is rung (subject to the setting of bell-style) and the original text is restored. An argument of n moves n positions forward in the list of matches; a negative argument may be used to move backward through the list. This command is intended to be bound to TAB, but is unbound by default.
c menu-complete-krd w
Identical to menu-complete, but moves backward through the list of possible completions, as if menu-complete had been given a negative argument. This command is unbound by default.
delete-char-or-list
Deletes the character under the cursor if not at the beginning or end of the line (like delete-char). If at the end of the line, behaves identically to possible-completions. This command is unbound by default.
complete-filename (M-/)
Attempt filename completion on the text before point.
possible-filename-completions (C-x /)
List the possible completions of the text before point, treating it as a filename.
complete-username (M-~)
Attempt completion on the text before point, treating it as a username.
possible-username-completions (C-x ~)
List the possible completions of the text before point, treating it as a username.
complete-variable (M-\$)
Attempt completion on the text before point, treating it as a shell variable.
possible-variable-completions (C-x \$)
List the possible completions of the text before point, treating it as a shell variable.
complete-hostname (M-@)
Attempt completion on the text before point, treating it as a hostname.
possible-hostname-completions (C-x @)
List the possible completions of the text before point, treating it as a hostname.
complete-command (M-!)
Attempt completion on the text before point, treating it as a command name. Command completion attempts to match the text against aliases, reserved words, shell functions, shell builtins, and finally executable filenames, in that order.
possible-command-completions (C-x !)
List the possible completions of the text before point, treating it as a command name.
dynamic-complete-history (M-TAB)
Attempt completion on the text before point, comparing the text against lines from the history list for possible completion matches.
dabbrev-expand
Attempt menu completion on the text before point, comparing the text against lines from the history list for possible completion matches.
complete-into-braces (M-{)
Perform filename completion and insert the list of possible completions enclosed within braces so the list is available to the shell (see Brace Expansion above).
Keyboard Macros

start-kbd-macro (C-x ()
Begin saving the characters typed into the current keyboard macro.
end-kbd-macro (C-x ))
Stop saving the characters typed into the current keyboard macro and store the definition.
call-last-kbd-macro (C-x e)
Re-execute the last keyboard macro defined, by making the characters in the macro appear as if typed at the keyboard.
Miscellaneous

re-read-init-file (C-x C-r)
Read in the contents of the inputrc file, and incorporate any bindings or variable assignments found there.
abort (C-g)
Abort the current editing command and ring the terminal's bell (subject to the setting of bell-style).
do-uppercase-version (M-a, M-b, M-x, ...)
If the metafied character x is lowercase, run the command that is bound to the corresponding uppercase character.
prefix-meta (ESC)
Metafy the next character typed. ESC f is equivalent to Meta-f.
undo (C-\_, C-x C-u)
Incremental undo, separately remembered for each line.
revert-line (M-r)
Undo all changes made to this line. This is like executing the undo command enough times to return the line to its initial state.
tilde-expand (M\&)
Perform tilde expansion on the current word.
set-mark (C-@, M-<space>)
Set the mark to the point. If a numeric argument is supplied, the mark is set to that position.
exchange-point-and-mark (C-x C-x)
Swap the point with the mark. The current cursor position is set to the saved position, and the old cursor position is saved as the mark.
character-search (C-])
A character is read and point is moved to the next occurrence of that character. A negative count searches for previous occurrences.
character-search-backward (M-C-])
A character is read and point is moved to the previous occurrence of that character. A negative count searches for subsequent occurrences.
skip-csi-sequence ()
Read enough characters to consume a multi-key sequence such as those defined for keys like Home and End. Such sequences begin with a Control Sequence Indicator (CSI), usually ESC-[. If this sequence is bound to "\[", keys producing such sequences will have no effect unless explicitly bound to a readline command, instead of inserting stray characters into the editing buffer. This is unbound by default, but usually bound to ESC-[.
insert-comment (M-\#)
Without a numeric argument, the value of the readline comment-begin variable is inserted at the beginning of the current line. If a numeric argument is supplied, this command acts as a toggle: if the characters at the beginning of the line do not match the value of comment-begin, the value is inserted, otherwise the characters in comment-begin are deleted from the beginning of the line. In either case, the line is accepted as if a newline had been typed. The default value of comment-begin causes this command to make the current line a shell comment. If a numeric argument causes the comment character to be removed, the line will be executed by the shell.
glob-complete-word (M-g)
The word before point is treated as a pattern for pathname expansion, with an asterisk implicitly appended. This pattern is used to generate a list of matching file names for possible completions.
glob-expand-word (C-x *)
The word before point is treated as a pattern for pathname expansion, and the list of matching file names is inserted, replacing the word. If a numeric argument is supplied, an asterisk is appended before pathname expansion.
glob-list-expansions (C-x g)
The list of expansions that would have been generated by glob-expand-word is displayed, and the line is redrawn. If a numeric argument is supplied, an asterisk is appended before pathname expansion.
dump-functions
Print all of the functions and their key bindings to the readline output stream. If a numeric argument is supplied, the output is formatted in such a way that it can be made part of an inputrc file.
dump-variables
Print all of the settable readline variables and their values to the readline output stream. If a numeric argument is supplied, the output is formatted in such a way that it can be made part of an inputrc file.
dump-macros
Print all of the readline key sequences bound to macros and the strings they output. If a numeric argument is supplied, the output is formatted in such a way that it can be made part of an inputrc file.
display-shell-version (C-x C-v)
Display version information about the current instance of bash.
Programmable Completion

When word completion is attempted for an argument to a command for which a completion specification (a compspec) has been defined using the complete builtin \hyperref[sec:shellbuiltincommands]{see SHELL BUILTIN COMMANDS below)}, the programmable completion facilities are invoked.
First, the command name is identified. If the command word is the empty string (completion attempted at the beginning of an empty line), any compspec defined with the -E option to complete is used. If a compspec has been defined for that command, the compspec is used to generate the list of possible completions for the word. If the command word is a full pathname, a compspec for the full pathname is searched for first. If no compspec is found for the full pathname, an attempt is made to find a compspec for the portion following the final slash. If those searches to not result in a compspec, any compspec defined with the -D option to complete is used as the default.

Once a compspec has been found, it is used to generate the list of matching words. If a compspec is not found, the default bash completion as described above under Completing is performed.

First, the actions specified by the compspec are used. Only matches which are prefixed by the word being completed are returned. When the -f or -d option is used for filename or directory name completion, the shell variable FIGNORE is used to filter the matches.

Any completions specified by a pathname expansion pattern to the -G option are generated next. The words generated by the pattern need not match the word being completed. The GLOBIGNORE shell variable is not used to filter the matches, but the FIGNORE variable is used.

Next, the string specified as the argument to the -W option is considered. The string is first split using the characters in the IFS special variable as delimiters. Shell quoting is honored. Each word is then expanded using brace expansion, tilde expansion, parameter and variable expansion, command substitution, and arithmetic expansion, as described above under EXPANSION. The results are split using the rules described above under Word Splitting. The results of the expansion are prefix-matched against the word being completed, and the matching words become the possible completions.

After these matches have been generated, any shell function or command specified with the -F and -C options is invoked. When the command or function is invoked, the COMP\_LINE, COMP\_POINT, COMP\_KEY, and COMP\_TYPE variables are assigned values as described above under Shell Variables. If a shell function is being invoked, the COMP\_WORDS and COMP\_CWORD variables are also set. When the function or command is invoked, the first argument is the name of the command whose arguments are being completed, the second argument is the word being completed, and the third argument is the word preceding the word being completed on the current command line. No filtering of the generated completions against the word being completed is performed; the function or command has complete freedom in generating the matches.

Any function specified with -F is invoked first. The function may use any of the shell facilities, including the compgen builtin described below, to generate the matches. It must put the possible completions in the COMPREPLY array variable.

Next, any command specified with the -C option is invoked in an environment equivalent to command substitution. It should print a list of completions, one per line, to the standard output. Backslash may be used to escape a newline, if necessary.

After all of the possible completions are generated, any filter specified with the -X option is applied to the list. The filter is a pattern as used for pathname expansion; a\& in the pattern is replaced with the text of the word being completed. A literal\& may be escaped with a backslash; the backslash is removed before attempting a match. Any completion that matches the pattern will be removed from the list. A leading ! negates the pattern; in this case any completion not matching the pattern will be removed.

Finally, any prefix and suffix specified with the -P and -S options are added to each member of the completion list, and the result is returned to the readline completion code as the list of possible completions.

If the previously-applied actions do not generate any matches, and the -o dirnames option was supplied to complete when the compspec was defined, directory name completion is attempted.

If the -o plusdirs option was supplied to complete when the compspec was defined, directory name completion is attempted and any matches are added to the results of the other actions.

By default, if a compspec is found, whatever it generates is returned to the completion code as the full set of possible completions. The default bash completions are not attempted, and the readline default of filename completion is disabled. If the -o bashdefault option was supplied to complete when the compspec was defined, the bash default completions are attempted if the compspec generates no matches. If the -o default option was supplied to complete when the compspec was defined, readline's default completion will be performed if the compspec (and, if attempted, the default bash completions) generate no matches.

When a compspec indicates that directory name completion is desired, the programmable completion functions force readline to append a slash to completed names which are symbolic links to directories, subject to the value of the mark-directories readline variable, regardless of the setting of the mark-symlinked-directories readline variable.

There is some support for dynamically modifying completions. This is most useful when used in combination with a default completion specified with complete -D. It's possible for shell functions executed as completion handlers to indicate that completion should be retried by returning an exit status of 124. If a shell function returns 124, and changes the compspec associated with the command on which completion is being attempted (supplied as the first argument when the function is executed), programmable completion restarts from the beginning, with an attempt to find a compspec for that command. This allows a set of completions to be built dynamically as completion is attempted, rather than being loaded all at once.

For instance, assuming that there is a library of compspecs, each kept in a file corresponding to the name of the command, the following default completion function would load completions dynamically:

\begin{lstlisting}


\_completion\_loader()
{

. "/etc/bash\_completion.d/\$1.sh" >/dev/null 2\&1\&\& return 124
}
\end{lstlisting}

complete -D -F \_completion\_loader