\section{Parameters}\label{sec:parameters}
A parameter is an entity that stores values. It can be a name, a number, or one of the special characters listed below under Special Parameters. A variable is a parameter denoted by a name. A variable has a value and zero or more attributes. Attributes are assigned using the declare builtin command (see declare below in SHELL BUILTIN COMMANDS).

A parameter is set if it has been assigned a value. The null string is a valid value. Once a variable is set, it may be unset only by using the unset builtin command \hyperref[sec:shellbuiltincommands]{see SHELL BUILTIN COMMANDS below)}.

A variable may be assigned to by a statement of the form

\begin{lstlisting}
name=[value]
\end{lstlisting}

If value is not given, the variable is assigned the null string. All values undergo tilde expansion, parameter and variable expansion, command substitution, arithmetic expansion, and quote removal \hyperref[sec:expansion]{(see EXPANSION below)}. If the variable has its integer attribute set, then value is evaluated as an arithmetic expression even if the \$\url{((...))} %bad syntax highlighting workaround.
expansion is not used \hyperref[sec:arithmeticexpansion]{(see Arithmetic Expansion below)}. Word splitting is not performed, with the exception of "\$@" as explained below under Special Parameters. Pathname expansion is not performed. Assignment statements may also appear as arguments to the alias, declare, typeset, export, readonly, and local builtin commands.
In the context where an assignment statement is assigning a value to a shell variable or array index, the += operator can be used to append to or add to the variable's previous value. When += is applied to a variable for which the integer attribute has been set, value is evaluated as an arithmetic expression and added to the variable's current value, which is also evaluated. When += is applied to an array variable using compound assignment \hyperref[sec:arrays]{(see Arrays below)}, the variable's value is not unset (as it is when using =), and new values are appended to the array beginning at one greater than the array's maximum index (for indexed arrays) or added as additional key-value pairs in an associative array. When applied to a string-valued variable, value is expanded and appended to the variable's value.

\subsection{Positional Parameters}\label{sec:positionalparameters}

A positional parameter is a parameter denoted by one or more digits, other than the single digit 0. Positional parameters are assigned from the shell's arguments when it is invoked, and may be reassigned using the set builtin command. Positional parameters may not be assigned to with assignment statements. The positional parameters are temporarily replaced when a shell function is executed \hyperref[sec:functions]{see FUNCTIONS below)}.
When a positional parameter consisting of more than a single digit is expanded, it must be enclosed in braces \hyperref[sec:expansion]{(see EXPANSION below)}.

\subsection{Special Parameters}\label{sec:specialparameters}

The shell treats several parameters specially. These parameters may only be referenced; assignment to them is not allowed.

\begin{longtable}{p{0.1\textwidth}p{0.85\textwidth}}

* &
Expands to the positional parameters, starting from one. When the expansion occurs within double quotes, it expands to a single word with the value of each parameter separated by the first character of the IFS special variable. That is, "\$*" is equivalent to "\$1c\$2c...", where c is the first character of the value of the IFS variable. If IFS is unset, the parameters are separated by spaces. If IFS is null, the parameters are joined without intervening separators. \\

@ &

Expands to the positional parameters, starting from one. When the expansion occurs within double quotes, each parameter expands to a separate word. That is, "\$@" is equivalent to "\$1" "\$2" ... If the double-quoted expansion occurs within a word, the expansion of the first parameter is joined with the beginning part of the original word, and the expansion of the last parameter is joined with the last part of the original word. When there are no positional parameters, "\$@" and \$@ expand to nothing (i.e., they are removed). \\

\# &

Expands to the number of positional parameters in decimal. \\

? &

Expands to the exit status of the most recently executed foreground pipeline. \\

- &

Expands to the current option flags as specified upon invocation, by the set builtin command, or those set by the shell itself (such as the -i option). \\

\$ &

Expands to the process ID of the shell. In a () subshell, it expands to the process ID of the current shell, not the subshell. \\

! &

Expands to the process ID of the most recently executed background (asynchronous) command. \\

0 &

Expands to the name of the shell or shell script. This is set at shell initialization. If bash is invoked with a file of commands, \$0 is set to the name of that file. If bash is started with the -c option, then \$0 is set to the first argument after the string to be executed, if one is present. Otherwise, it is set to the file name used to invoke bash, as given by argument zero. \\

\_ &

At shell startup, set to the absolute pathname used to invoke the shell or shell script being executed as passed in the environment or argument list. Subsequently, expands to the last argument to the previous command, after expansion. Also set to the full pathname used to invoke each command executed and placed in the environment exported to that command. When checking mail, this parameter holds the name of the mail file currently being checked. \\

\end{longtable}