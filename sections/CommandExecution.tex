\section{Command Execution}\label{sec:commandexecution}
After a command has been split into words, if it results in a simple command and an optional list of arguments, the following actions are taken.

If the command name contains no slashes, the shell attempts to locate it. If there exists a shell function by that name, that function is invoked as described above in FUNCTIONS. If the name does not match a function, the shell searches for it in the list of shell builtins. If a match is found, that builtin is invoked.

If the name is neither a shell function nor a builtin, and contains no slashes, bash searches each element of the PATH for a directory containing an executable file by that name. Bash uses a hash table to remember the full pathnames of executable files (see hash under SHELL BUILTIN COMMANDS below)}. A full search of the directories in PATH is performed only if the command is not found in the hash table. If the search is unsuccessful, the shell searches for a defined shell function named command\_not\_found\_handle. If that function exists, it is invoked with the original command and the original command's arguments as its arguments, and the function's exit status becomes the exit status of the shell. If that function is not defined, the shell prints an error message and returns an exit status of 127.

If the search is successful, or if the command name contains one or more slashes, the shell executes the named program in a separate execution environment. Argument 0 is set to the name given, and the remaining arguments to the command are set to the arguments given, if any.

If this execution fails because the file is not in executable format, and the file is not a directory, it is assumed to be a shell script, a file containing shell commands. A subshell is spawned to execute it. This subshell reinitializes itself, so that the effect is as if a new shell had been invoked to handle the script, with the exception that the locations of commands remembered by the parent (see hash below under SHELL BUILTIN COMMANDS) are retained by the child.

If the program is a file beginning with \#!, the remainder of the first line specifies an interpreter for the program. The shell executes the specified interpreter on operating systems that do not handle this executable format themselves. The arguments to the interpreter consist of a single optional argument following the interpreter name on the first line of the program, followed by the name of the program, followed by the command arguments, if any.