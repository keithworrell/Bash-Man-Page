\section{Shell Builtin Commands}\label{sec:shellbuiltincommands}
Unless otherwise noted, each builtin command documented in this section as accepting options preceded by - accepts -- to signify the end of the options. The :, true, false, and test builtins do not accept options and do not treat -- specially. The exit, logout, break, continue, let, and shift builtins accept and process arguments beginning with - without requiring --. Other builtins that accept arguments but are not specified as accepting options interpret arguments beginning with - as invalid options and require -- to prevent this interpretation.

: [arguments]
No effect; the command does nothing beyond expanding arguments and performing any specified redirections. A zero exit code is returned.
. filename [arguments]
source filename [arguments]
Read and execute commands from filename in the current shell environment and return the exit status of the last command executed from filename. If filename does not contain a slash, file names in PATH are used to find the directory containing filename. The file searched for in PATH need not be executable. When bash is not in posix mode, the current directory is searched if no file is found in PATH. If the sourcepath option to the shopt builtin command is turned off, the PATH is not searched. If any arguments are supplied, they become the positional parameters when filename is executed. Otherwise the positional parameters are unchanged. The return status is the status of the last command exited within the script (0 if no commands are executed), and false if filename is not found or cannot be read.
alias [-p] [name[=value] ...]
Alias with no arguments or with the -p option prints the list of aliases in the form alias name=value on standard output. When arguments are supplied, an alias is defined for each name whose value is given. A trailing space in value causes the next word to be checked for alias substitution when the alias is expanded. For each name in the argument list for which no value is supplied, the name and value of the alias is printed. Alias returns true unless a name is given for which no alias has been defined.
bg [jobspec ...]
Resume each suspended job jobspec in the background, as if it had been started with\&. If jobspec is not present, the shell's notion of the current job is used. bg jobspec returns 0 unless run when job control is disabled or, when run with job control enabled, any specified jobspec was not found or was started without job control.
bind [-m keymap] [-lpsvPSV]
bind [-m keymap] [-q function] [-u function] [-r keyseq]
bind [-m keymap] -f filename
bind [-m keymap] -x keyseq:shell-command
bind [-m keymap] keyseq:function-name
bind readline-command
Display current readline key and function bindings, bind a key sequence to a readline function or macro, or set a readline variable. Each non-option argument is a command as it would appear in .inputrc, but each binding or command must be passed as a separate argument; e.g., '"\C-x\C-r": re-read-init-file'. Options, if supplied, have the following meanings:
-m keymap
Use keymap as the keymap to be affected by the subsequent bindings. Acceptable keymap names are emacs, emacs-standard, emacs-meta, emacs-ctlx, vi, vi-move, vi-command, and vi-insert. vi is equivalent to vi-command; emacs is equivalent to emacs-standard.

-l
List the names of all readline functions.

-p

Display readline function names and bindings in such a way that they can be re-read.

-P

List current readline function names and bindings.

-s

Display readline key sequences bound to macros and the strings they output in such a way that they can be re-read.

-S

Display readline key sequences bound to macros and the strings they output.

-v

Display readline variable names and values in such a way that they can be re-read.

-V

List current readline variable names and values.

-f filename
Read key bindings from filename.

-q function
Query about which keys invoke the named function.

-u function
Unbind all keys bound to the named function.

-r keyseq
Remove any current binding for keyseq.

-x keyseq:shell-command
Cause shell-command to be executed whenever keyseq is entered. When shell-command is executed, the shell sets the READLINE\_LINE variable to the contents of the readline line buffer and the READLINE\_POINT variable to the current location of the insertion point. If the executed command changes the value of READLINE\_LINE or READLINE\_POINT, those new values will be reflected in the editing state.

The return value is 0 unless an unrecognized option is given or an error occurred.
break [n]
Exit from within a for, while, until, or select loop. If n is specified, break n levels. n must be ≥ 1. If n is greater than the number of enclosing loops, all enclosing loops are exited. The return value is non-zero when n is ≤ 0; Otherwise, break returns 0 value.
builtin shell-builtin [arguments]
Execute the specified shell builtin, passing it arguments, and return its exit status. This is useful when defining a function whose name is the same as a shell builtin, retaining the functionality of the builtin within the function. The cd builtin is commonly redefined this way. The return status is false if shell-builtin is not a shell builtin command.
caller [expr]
Returns the context of any active subroutine call (a shell function or a script executed with the . or source builtins. Without expr, caller displays the line number and source filename of the current subroutine call. If a non-negative integer is supplied as expr, caller displays the line number, subroutine name, and source file corresponding to that position in the current execution call stack. This extra information may be used, for example, to print a stack trace. The current frame is frame 0. The return value is 0 unless the shell is not executing a subroutine call or expr does not correspond to a valid position in the call stack.
cd [-L|-P] [dir]
Change the current directory to dir. The variable HOME is the default dir. The variable CDPATH defines the search path for the directory containing dir. Alternative directory names in CDPATH are separated by a colon (:). A null directory name in CDPATH is the same as the current directory, i.e., ''.''. If dir begins with a slash (/), then CDPATH is not used. The -P option says to use the physical directory structure instead of following symbolic links (see also the -P option to the set builtin command); the -L option forces symbolic links to be followed. An argument of - is equivalent to \$OLDPWD. If a non-empty directory name from CDPATH is used, or if - is the first argument, and the directory change is successful, the absolute pathname of the new working directory is written to the standard output. The return value is true if the directory was successfully changed; false otherwise.
command [-pVv] command [arg ...]
Run command with args suppressing the normal shell function lookup. Only builtin commands or commands found in the PATH are executed. If the -p option is given, the search for command is performed using a default value for PATH that is guaranteed to find all of the standard utilities. If either the -V or -v option is supplied, a description of command is printed. The -v option causes a single word indicating the command or file name used to invoke command to be displayed; the -V option produces a more verbose description. If the -V or -v option is supplied, the exit status is 0 if command was found, and 1 if not. If neither option is supplied and an error occurred or command cannot be found, the exit status is 127. Otherwise, the exit status of the command builtin is the exit status of command.
compgen [option] [word]
Generate possible completion matches for word according to the options, which may be any option accepted by the complete builtin with the exception of -p and -r, and write the matches to the standard output. When using the -F or -C options, the various shell variables set by the programmable completion facilities, while available, will not have useful values.
The matches will be generated in the same way as if the programmable completion code had generated them directly from a completion specification with the same flags. If word is specified, only those completions matching word will be displayed.

The return value is true unless an invalid option is supplied, or no matches were generated.

complete [-abcdefgjksuv] [-o comp-option] [-DE] [-A action] [-G globpat] [-W wordlist] [-F function] [-C command]
[-X filterpat] [-P prefix] [-S suffix] name [name ...]
complete -pr [-DE] [name ...]
Specify how arguments to each name should be completed. If the -p option is supplied, or if no options are supplied, existing completion specifications are printed in a way that allows them to be reused as input. The -r option removes a completion specification for each name, or, if no names are supplied, all completion specifications. The -D option indicates that the remaining options and actions should apply to the ''default'' command completion; that is, completion attempted on a command for which no completion has previously been defined. The -E option indicates that the remaining options and actions should apply to ''empty'' command completion; that is, completion attempted on a blank line.
The process of applying these completion specifications when word completion is attempted is described above under Programmable Completion.

Other options, if specified, have the following meanings. The arguments to the -G, -W, and -X options (and, if necessary, the -P and -S options) should be quoted to protect them from expansion before the complete builtin is invoked.

-o comp-option
The comp-option controls several aspects of the compspec's behavior beyond the simple generation of completions. comp-option may be one of:

bashdefault

Perform the rest of the default bash completions if the compspec generates no matches.

default

Use readline's default filename completion if the compspec generates no matches.

dirnames

Perform directory name completion if the compspec generates no matches.

filenames

Tell readline that the compspec generates filenames, so it can perform any filename-specific processing (like adding a slash to directory names, quoting special characters, or suppressing trailing spaces). Intended to be used with shell functions.

nospace

Tell readline not to append a space (the default) to words completed at the end of the line.

plusdirs

After any matches defined by the compspec are generated, directory name completion is attempted and any matches are added to the results of the other actions.

-A action
The action may be one of the following to generate a list of possible completions:

alias

Alias names. May also be specified as -a.

arrayvar

Array variable names.

binding

Readline key binding names.

builtin

Names of shell builtin commands. May also be specified as -b.

command

Command names. May also be specified as -c.

directory

Directory names. May also be specified as -d.

disabled

Names of disabled shell builtins.

enabled

Names of enabled shell builtins.

export

Names of exported shell variables. May also be specified as -e.

file

File names. May also be specified as -f.

function

Names of shell functions.

group

Group names. May also be specified as -g.

helptopic

Help topics as accepted by the help builtin.

hostname

Hostnames, as taken from the file specified by the HOSTFILE shell variable.

job

Job names, if job control is active. May also be specified as -j.

keyword

Shell reserved words. May also be specified as -k.

running

Names of running jobs, if job control is active.

service

Service names. May also be specified as -s.

setopt

Valid arguments for the -o option to the set builtin.

shopt

Shell option names as accepted by the shopt builtin.

signal

Signal names.

stopped

Names of stopped jobs, if job control is active.

user

User names. May also be specified as -u.

variable

Names of all shell variables. May also be specified as -v.

-G globpat
The pathname expansion pattern globpat is expanded to generate the possible completions.

-W wordlist
The wordlist is split using the characters in the IFS special variable as delimiters, and each resultant word is expanded. The possible completions are the members of the resultant list which match the word being completed.

-C command
command is executed in a subshell environment, and its output is used as the possible completions.

-F function
The shell function function is executed in the current shell environment. When it finishes, the possible completions are retrieved from the value of the COMPREPLY array variable.

-X filterpat
filterpat is a pattern as used for pathname expansion. It is applied to the list of possible completions generated by the preceding options and arguments, and each completion matching filterpat is removed from the list. A leading ! in filterpat negates the pattern; in this case, any completion not matching filterpat is removed.

-P prefix
prefix is added at the beginning of each possible completion after all other options have been applied.

-S suffix
suffix is appended to each possible completion after all other options have been applied.

The return value is true unless an invalid option is supplied, an option other than -p or -r is supplied without a name argument, an attempt is made to remove a completion specification for a name for which no specification exists, or an error occurs adding a completion specification.
compopt [-o option] [-DE] [+o option] [name]
Modify completion options for each name according to the options, or for the currently-execution completion if no names are supplied. If no options are given, display the completion options for each name or the current completion. The possible values of option are those valid for the complete builtin described above. The -D option indicates that the remaining options should apply to the ''default'' command completion; that is, completion attempted on a command for which no completion has previously been defined. The -E option indicates that the remaining options should apply to ''empty'' command completion; that is, completion attempted on a blank line.
The return value is true unless an invalid option is supplied, an attempt is made to modify the options for a name for which no completion specification exists, or an output error occurs.
continue [n]
Resume the next iteration of the enclosing for, while, until, or select loop. If n is specified, resume at the nth enclosing loop. n must be ≥ 1. If n is greater than the number of enclosing loops, the last enclosing loop (the ''top-level'' loop) is resumed. When continue is executed inside of loop, the return value is non-zero when n is ≤ 0; Otherwise, continue returns 0 value. When continue is executed outside of loop, the return value is 0.
declare [-aAfFilrtux] [-p] [name[=value] ...]
typeset [-aAfFilrtux] [-p] [name[=value] ...]
Declare variables and/or give them attributes. If no names are given then display the values of variables. The -p option will display the attributes and values of each name. When -p is used with name arguments, additional options are ignored. When -p is supplied without name arguments, it will display the attributes and values of all variables having the attributes specified by the additional options. If no other options are supplied with -p, declare will display the attributes and values of all shell variables. The -f option will restrict the display to shell functions. The -F option inhibits the display of function definitions; only the function name and attributes are printed. If the extdebug shell option is enabled using shopt, the source file name and line number where the function is defined are displayed as well. The -F option implies -f. The following options can be used to restrict output to variables with the specified attribute or to give variables attributes:
-a
Each name is an indexed array variable (see Arrays above).

-A

Each name is an associative array variable (see Arrays above).

-f

Use function names only.

-i

The variable is treated as an integer; arithmetic evaluation \hyperref[sec:arithmeticevaluation]{see ARITHMETIC EVALUATION above) is performed when the variable is assigned a value.

-l

When the variable is assigned a value, all upper-case characters are converted to lower-case. The upper-case attribute is disabled.

-r

Make names readonly. These names cannot then be assigned values by subsequent assignment statements or unset.

-t

Give each name the trace attribute. Traced functions inherit the DEBUG and RETURN traps from the calling shell. The trace attribute has no special meaning for variables.

-u

When the variable is assigned a value, all lower-case characters are converted to upper-case. The lower-case attribute is disabled.

-x

Mark names for export to subsequent commands via the environment.

Using '+' instead of '-' turns off the attribute instead, with the exceptions that +a may not be used to destroy an array variable and +r will not remove the readonly attribute. When used in a function, makes each name local, as with the local command. If a variable name is followed by =value, the value of the variable is set to value. The return value is 0 unless an invalid option is encountered, an attempt is made to define a function using ''-f foo=bar'', an attempt is made to assign a value to a readonly variable, an attempt is made to assign a value to an array variable without using the compound assignment syntax (see Arrays above), one of the names is not a valid shell variable name, an attempt is made to turn off readonly status for a readonly variable, an attempt is made to turn off array status for an array variable, or an attempt is made to display a non-existent function with -f.
dirs [+n] [-n] [-cplv]
Without options, displays the list of currently remembered directories. The default display is on a single line with directory names separated by spaces. Directories are added to the list with the pushd command; the popd command removes entries from the list.
+n
Displays the nth entry counting from the left of the list shown by dirs when invoked without options, starting with zero.

-n

Displays the nth entry counting from the right of the list shown by dirs when invoked without options, starting with zero.

-c

Clears the directory stack by deleting all of the entries.

-l

Produces a longer listing; the default listing format uses a tilde to denote the home directory.

-p

Print the directory stack with one entry per line.

-v

Print the directory stack with one entry per line, prefixing each entry with its index in the stack.

The return value is 0 unless an invalid option is supplied or n indexes beyond the end of the directory stack.
disown [-ar] [-h] [jobspec ...]
Without options, each jobspec is removed from the table of active jobs. If jobspec is not present, and neither -a nor -r is supplied, the shell's notion of the current job is used. If the -h option is given, each jobspec is not removed from the table, but is marked so that SIGHUP is not sent to the job if the shell receives a SIGHUP. If no jobspec is present, and neither the -a nor the -r option is supplied, the current job is used. If no jobspec is supplied, the -a option means to remove or mark all jobs; the -r option without a jobspec argument restricts operation to running jobs. The return value is 0 unless a jobspec does not specify a valid job.
echo [-neE] [arg ...]
Output the args, separated by spaces, followed by a newline. The return status is always 0. If -n is specified, the trailing newline is suppressed. If the -e option is given, interpretation of the following backslash-escaped characters is enabled. The -E option disables the interpretation of these escape characters, even on systems where they are interpreted by default. The xpg\_echo shell option may be used to dynamically determine whether or not echo expands these escape characters by default. echo does not interpret -- to mean the end of options. echo interprets the following escape sequences:
\a
alert (bell)

\b

backspace

\c

suppress further output

\e

an escape character

\f

form feed

\n

new line

\r

carriage return

\t

horizontal tab

\v

vertical tab

\\

backslash

\0nnn

the eight-bit character whose value is the octal value nnn (zero to three octal digits)

\xHH

the eight-bit character whose value is the hexadecimal value HH (one or two hex digits)

enable [-a] [-dnps] [-f filename] [name ...]
Enable and disable builtin shell commands. Disabling a builtin allows a disk command which has the same name as a shell builtin to be executed without specifying a full pathname, even though the shell normally searches for builtins before disk commands. If -n is used, each name is disabled; otherwise, names are enabled. For example, to use the test binary found via the PATH instead of the shell builtin version, run ''enable -n test''. The -f option means to load the new builtin command name from shared object filename, on systems that support dynamic loading. The -d option will delete a builtin previously loaded with -f. If no name arguments are given, or if the -p option is supplied, a list of shell builtins is printed. With no other option arguments, the list consists of all enabled shell builtins. If -n is supplied, only disabled builtins are printed. If -a is supplied, the list printed includes all builtins, with an indication of whether or not each is enabled. If -s is supplied, the output is restricted to the POSIX special builtins. The return value is 0 unless a name is not a shell builtin or there is an error loading a new builtin from a shared object.
eval [arg ...]
The args are read and concatenated together into a single command. This command is then read and executed by the shell, and its exit status is returned as the value of eval. If there are no args, or only null arguments, eval returns 0.
exec [-cl] [-a name] [command [arguments]]
If command is specified, it replaces the shell. No new process is created. The arguments become the arguments to command. If the -l option is supplied, the shell places a dash at the beginning of the zeroth argument passed to command. This is what login(1) does. The -c option causes command to be executed with an empty environment. If -a is supplied, the shell passes name as the zeroth argument to the executed command. If command cannot be executed for some reason, a non-interactive shell exits, unless the shell option execfail is enabled, in which case it returns failure. An interactive shell returns failure if the file cannot be executed. If command is not specified, any redirections take effect in the current shell, and the return status is 0. If there is a redirection error, the return status is 1.
exit [n]
Cause the shell to exit with a status of n. If n is omitted, the exit status is that of the last command executed. A trap on EXIT is executed before the shell terminates.
export [-fn] [name[=word]] ...
export -p
The supplied names are marked for automatic export to the environment of subsequently executed commands. If the -f option is given, the names refer to functions. If no names are given, or if the -p option is supplied, a list of all names that are exported in this shell is printed. The -n option causes the export property to be removed from each name. If a variable name is followed by =word, the value of the variable is set to word. export returns an exit status of 0 unless an invalid option is encountered, one of the names is not a valid shell variable name, or -f is supplied with a name that is not a function.
fc [-e ename] [-lnr] [first] [last]
fc -s [pat=rep] [cmd]
Fix Command. In the first form, a range of commands from first to last is selected from the history list. First and last may be specified as a string (to locate the last command beginning with that string) or as a number (an index into the history list, where a negative number is used as an offset from the current command number). If last is not specified it is set to the current command for listing (so that ''fc -l -10'' prints the last 10 commands) and to first otherwise. If first is not specified it is set to the previous command for editing and -16 for listing.
The -n option suppresses the command numbers when listing. The -r option reverses the order of the commands. If the -l option is given, the commands are listed on standard output. Otherwise, the editor given by ename is invoked on a file containing those commands. If ename is not given, the value of the FCEDIT variable is used, and the value of EDITOR if FCEDIT is not set. If neither variable is set, vi is used. When editing is complete, the edited commands are echoed and executed.

In the second form, command is re-executed after each instance of pat is replaced by rep. A useful alias to use with this is ''r="fc -s"'', so that typing ''r cc'' runs the last command beginning with ''cc'' and typing ''r'' re-executes the last command.

If the first form is used, the return value is 0 unless an invalid option is encountered or first or last specify history lines out of range. If the -e option is supplied, the return value is the value of the last command executed or failure if an error occurs with the temporary file of commands. If the second form is used, the return status is that of the command re-executed, unless cmd does not specify a valid history line, in which case fc returns failure.

fg [jobspec]
Resume jobspec in the foreground, and make it the current job. If jobspec is not present, the shell's notion of the current job is used. The return value is that of the command placed into the foreground, or failure if run when job control is disabled or, when run with job control enabled, if jobspec does not specify a valid job or jobspec specifies a job that was started without job control.
getopts optstring name [args]
getopts is used by shell procedures to parse positional parameters. optstring contains the option characters to be recognized; if a character is followed by a colon, the option is expected to have an argument, which should be separated from it by white space. The colon and question mark characters may not be used as option characters. Each time it is invoked, getopts places the next option in the shell variable name, initializing name if it does not exist, and the index of the next argument to be processed into the variable OPTIND. OPTIND is initialized to 1 each time the shell or a shell script is invoked. When an option requires an argument, getopts places that argument into the variable OPTARG. The shell does not reset OPTIND automatically; it must be manually reset between multiple calls to getopts within the same shell invocation if a new set of parameters is to be used.
When the end of options is encountered, getopts exits with a return value greater than zero. OPTIND is set to the index of the first non-option argument, and name is set to ?.

getopts normally parses the positional parameters, but if more arguments are given in args, getopts parses those instead.

getopts can report errors in two ways. If the first character of optstring is a colon, silent error reporting is used. In normal operation diagnostic messages are printed when invalid options or missing option arguments are encountered. If the variable OPTERR is set to 0, no error messages will be displayed, even if the first character of optstring is not a colon.

If an invalid option is seen, getopts places ? into name and, if not silent, prints an error message and unsets OPTARG. If getopts is silent, the option character found is placed in OPTARG and no diagnostic message is printed.

If a required argument is not found, and getopts is not silent, a question mark (?) is placed in name, OPTARG is unset, and a diagnostic message is printed. If getopts is silent, then a colon (:) is placed in name and OPTARG is set to the option character found.

getopts returns true if an option, specified or unspecified, is found. It returns false if the end of options is encountered or an error occurs.

hash [-lr] [-p filename] [-dt] [name]
For each name, the full file name of the command is determined by searching the directories in \$PATH and remembered. If the -p option is supplied, no path search is performed, and filename is used as the full file name of the command. The -r option causes the shell to forget all remembered locations. The -d option causes the shell to forget the remembered location of each name. If the -t option is supplied, the full pathname to which each name corresponds is printed. If multiple name arguments are supplied with -t, the name is printed before the hashed full pathname. The -l option causes output to be displayed in a format that may be reused as input. If no arguments are given, or if only -l is supplied, information about remembered commands is printed. The return status is true unless a name is not found or an invalid option is supplied.
help [-dms] [pattern]
Display helpful information about builtin commands. If pattern is specified, help gives detailed help on all commands matching pattern; otherwise help for all the builtins and shell control structures is printed.
-d
Display a short description of each pattern

-m

Display the description of each pattern in a manpage-like format

-s

Display only a short usage synopsis for each pattern

The return status is 0 unless no command matches pattern.
history [n]
history -c
history -d offset
history -anrw [filename]
history -p arg [arg ...]
history -s arg [arg ...]
With no options, display the command history list with line numbers. Lines listed with a * have been modified. An argument of n lists only the last n lines. If the shell variable HISTTIMEFORMAT is set and not null, it is used as a format string for strftime(3) to display the time stamp associated with each displayed history entry. No intervening blank is printed between the formatted time stamp and the history line. If filename is supplied, it is used as the name of the history file; if not, the value of HISTFILE is used. Options, if supplied, have the following meanings:
-c
Clear the history list by deleting all the entries.

-d offset
Delete the history entry at position offset.

-a
Append the ''new'' history lines (history lines entered since the beginning of the current bash session) to the history file.

-n

Read the history lines not already read from the history file into the current history list. These are lines appended to the history file since the beginning of the current bash session.

-r

Read the contents of the history file and use them as the current history.

-w

Write the current history to the history file, overwriting the history file's contents.

-p

Perform history substitution on the following args and display the result on the standard output. Does not store the results in the history list. Each arg must be quoted to disable normal history expansion.

-s

Store the args in the history list as a single entry. The last command in the history list is removed before the args are added.

If the HISTTIMEFORMAT variable is set, the time stamp information associated with each history entry is written to the history file, marked with the history comment character. When the history file is read, lines beginning with the history comment character followed immediately by a digit are interpreted as timestamps for the previous history line. The return value is 0 unless an invalid option is encountered, an error occurs while reading or writing the history file, an invalid offset is supplied as an argument to -d, or the history expansion supplied as an argument to -p fails.
jobs [-lnprs] [ jobspec ... ]
jobs -x command [ args ... ]
The first form lists the active jobs. The options have the following meanings:
-l
List process IDs in addition to the normal information.

-p

List only the process ID of the job's process group leader.

-n

Display information only about jobs that have changed status since the user was last notified of their status.

-r

Restrict output to running jobs.

-s

Restrict output to stopped jobs.

If jobspec is given, output is restricted to information about that job. The return status is 0 unless an invalid option is encountered or an invalid jobspec is supplied.
If the -x option is supplied, jobs replaces any jobspec found in command or args with the corresponding process group ID, and executes command passing it args, returning its exit status.

kill [-s sigspec | -n signum | -sigspec] [pid | jobspec] ...
kill -l [sigspec | exit\_status]
Send the signal named by sigspec or signum to the processes named by pid or jobspec. sigspec is either a case-insensitive signal name such as SIGKILL (with or without the SIG prefix) or a signal number; signum is a signal number. If sigspec is not present, then SIGTERM is assumed. An argument of -l lists the signal names. If any arguments are supplied when -l is given, the names of the signals corresponding to the arguments are listed, and the return status is 0. The exit\_status argument to -l is a number specifying either a signal number or the exit status of a process terminated by a signal. kill returns true if at least one signal was successfully sent, or false if an error occurs or an invalid option is encountered.
let arg [arg ...]
Each arg is an arithmetic expression to be evaluated \hyperref[sec:arithmeticevaluation]{see ARITHMETIC EVALUATION above). If the last arg evaluates to 0, let returns 1; 0 is returned otherwise.
local [option] [name[=value] ...]
For each argument, a local variable named name is created, and assigned value. The option can be any of the options accepted by declare. When local is used within a function, it causes the variable name to have a visible scope restricted to that function and its children. With no operands, local writes a list of local variables to the standard output. It is an error to use local when not within a function. The return status is 0 unless local is used outside a function, an invalid name is supplied, or name is a readonly variable.
logout
Exit a login shell.

mapfile [-n count] [-O origin] [-s count] [-t] [-u fd] [-C callback] [-c quantum] [array]
readarray [-n count] [-O origin] [-s count] [-t] [-u fd] [-C callback] [-c quantum] [array]
Read lines from the standard input into the indexed array variable array, or from file descriptor fd if the -u option is supplied. The variable MAPFILE is the default array. Options, if supplied, have the following meanings:
-n
Copy at most count lines. If count is 0, all lines are copied.

-O

Begin assigning to array at index origin. The default index is 0.

-s

Discard the first count lines read.

-t

Remove a trailing newline from each line read.

-u

Read lines from file descriptor fd instead of the standard input.

-C

Evaluate callback each time quantum lines are read. The -c option specifies quantum.

-c

Specify the number of lines read between each call to callback.

If -C is specified without -c, the default quantum is 5000. When callback is evaluated, it is supplied the index of the next array element to be assigned as an additional argument. callback is evaluated after the line is read but before the array element is assigned.
If not supplied with an explicit origin, mapfile will clear array before assigning to it.

mapfile returns successfully unless an invalid option or option argument is supplied, array is invalid or unassignable, or if array is not an indexed array.

popd [-n] [+n] [-n]
Removes entries from the directory stack. With no arguments, removes the top directory from the stack, and performs a cd to the new top directory. Arguments, if supplied, have the following meanings:
-n
Suppresses the normal change of directory when removing directories from the stack, so that only the stack is manipulated.

+n

Removes the nth entry counting from the left of the list shown by dirs, starting with zero. For example: ''popd +0'' removes the first directory, ''popd +1'' the second.

-n

Removes the nth entry counting from the right of the list shown by dirs, starting with zero. For example: ''popd -0'' removes the last directory, ''popd -1'' the next to last.

If the popd command is successful, a dirs is performed as well, and the return status is 0. popd returns false if an invalid option is encountered, the directory stack is empty, a non-existent directory stack entry is specified, or the directory change fails.
printf [-v var] format [arguments]
Write the formatted arguments to the standard output under the control of the format. The format is a character string which contains three types of objects: plain characters, which are simply copied to standard output, character escape sequences, which are converted and copied to the standard output, and format specifications, each of which causes printing of the next successive argument. In addition to the standard printf(1) formats, \%b causes printf to expand backslash escape sequences in the corresponding argument (except that \c terminates output, backslashes in \', \", and \? are not removed, and octal escapes beginning with \0 may contain up to four digits), and \%q causes printf to output the corresponding argument in a format that can be reused as shell input.
The -v option causes the output to be assigned to the variable var rather than being printed to the standard output.

The format is reused as necessary to consume all of the arguments. If the format requires more arguments than are supplied, the extra format specifications behave as if a zero value or null string, as appropriate, had been supplied. The return value is zero on success, non-zero on failure.

pushd [-n] [+n] [-n]
pushd [-n] [dir]
Adds a directory to the top of the directory stack, or rotates the stack, making the new top of the stack the current working directory. With no arguments, exchanges the top two directories and returns 0, unless the directory stack is empty. Arguments, if supplied, have the following meanings:
-n
Suppresses the normal change of directory when adding directories to the stack, so that only the stack is manipulated.

+n

Rotates the stack so that the nth directory (counting from the left of the list shown by dirs, starting with zero) is at the top.

-n

Rotates the stack so that the nth directory (counting from the right of the list shown by dirs, starting with zero) is at the top.

dir

Adds dir to the directory stack at the top, making it the new current working directory.

If the pushd command is successful, a dirs is performed as well. If the first form is used, pushd returns 0 unless the cd to dir fails. With the second form, pushd returns 0 unless the directory stack is empty, a non-existent directory stack element is specified, or the directory change to the specified new current directory fails.
pwd [-LP]
Print the absolute pathname of the current working directory. The pathname printed contains no symbolic links if the -P option is supplied or the -o physical option to the set builtin command is enabled. If the -L option is used, the pathname printed may contain symbolic links. The return status is 0 unless an error occurs while reading the name of the current directory or an invalid option is supplied.
read [-ers] [-a aname] [-d delim] [-i text] [-n nchars] [-N nchars] [-p prompt] [-t timeout] [-u fd] [name ...]
One line is read from the standard input, or from the file descriptor fd supplied as an argument to the -u option, and the first word is assigned to the first name, the second word to the second name, and so on, with leftover words and their intervening separators assigned to the last name. If there are fewer words read from the input stream than names, the remaining names are assigned empty values. The characters in IFS are used to split the line into words. The backslash character (\) may be used to remove any special meaning for the next character read and for line continuation. Options, if supplied, have the following meanings:
-a aname
The words are assigned to sequential indices of the array variable aname, starting at 0. aname is unset before any new values are assigned. Other name arguments are ignored.

-d delim
The first character of delim is used to terminate the input line, rather than newline.

-e
If the standard input is coming from a terminal, readline \hyperref[sec:readline]{see READLINE above) is used to obtain the line. Readline uses the current (or default, if line editing was not previously active) editing settings.

-i text
If readline is being used to read the line, text is placed into the editing buffer before editing begins.

-n nchars
read returns after reading nchars characters rather than waiting for a complete line of input, but honor a delimiter if fewer than nchars characters are read before the delimiter.

-N nchars
read returns after reading exactly nchars characters rather than waiting for a complete line of input, unless EOF is encountered or read times out. Delimiter characters encountered in the input are not treated specially and do not cause read to return until nchars characters are read.

-p prompt
Display prompt on standard error, without a trailing newline, before attempting to read any input. The prompt is displayed only if input is coming from a terminal.

-r
Backslash does not act as an escape character. The backslash is considered to be part of the line. In particular, a backslash-newline pair may not be used as a line continuation.

-s

Silent mode. If input is coming from a terminal, characters are not echoed.

-t timeout
Cause read to time out and return failure if a complete line of input is not read within timeout seconds. timeout may be a decimal number with a fractional portion following the decimal point. This option is only effective if read is reading input from a terminal, pipe, or other special file; it has no effect when reading from regular files. If timeout is 0, read returns success if input is available on the specified file descriptor, failure otherwise. The exit status is greater than 128 if the timeout is exceeded.

-u fd
Read input from file descriptor fd.

If no names are supplied, the line read is assigned to the variable REPLY. The return code is zero, unless end-of-file is encountered, read times out (in which case the return code is greater than 128), or an invalid file descriptor is supplied as the argument to -u.
readonly [-aApf] [name[=word] ...]
The given names are marked readonly; the values of these names may not be changed by subsequent assignment. If the -f option is supplied, the functions corresponding to the names are so marked. The -a option restricts the variables to indexed arrays; the -A option restricts the variables to associative arrays. If no name arguments are given, or if the -p option is supplied, a list of all readonly names is printed. The -p option causes output to be displayed in a format that may be reused as input. If a variable name is followed by =word, the value of the variable is set to word. The return status is 0 unless an invalid option is encountered, one of the names is not a valid shell variable name, or -f is supplied with a name that is not a function.
return [n]
Causes a function to exit with the return value specified by n. If n is omitted, the return status is that of the last command executed in the function body. If used outside a function, but during execution of a script by the . (source) command, it causes the shell to stop executing that script and return either n or the exit status of the last command executed within the script as the exit status of the script. If used outside a function and not during execution of a script by ., the return status is false. Any command associated with the RETURN trap is executed before execution resumes after the function or script.
set [--abefhkmnptuvxBCEHPT] [-o option] [arg ...]
set [+abefhkmnptuvxBCEHPT] [+o option] [arg ...]
Without options, the name and value of each shell variable are displayed in a format that can be reused as input for setting or resetting the currently-set variables. Read-only variables cannot be reset. In posix mode, only shell variables are listed. The output is sorted according to the current locale. When options are specified, they set or unset shell attributes. Any arguments remaining after option processing are treated as values for the positional parameters and are assigned, in order, to \$1, \$2, ... \$n. Options, if specified, have the following meanings:
-a
Automatically mark variables and functions which are modified or created for export to the environment of subsequent commands.

-b

Report the status of terminated background jobs immediately, rather than before the next primary prompt. This is effective only when job control is enabled.

-e

Exit immediately if a pipeline (which may consist of a single simple command), a subshell command enclosed in parentheses, or one of the commands executed as part of a command list enclosed by braces \hyperref[sec:shellgrammar]{see SHELL GRAMMAR above) exits with a non-zero status. The shell does not exit if the command that fails is part of the command list immediately following a while or until keyword, part of the test following the if or elif reserved words, part of any command executed in a\&\& or || list except the command following the final\&\& or ||, any command in a pipeline but the last, or if the command's return value is being inverted with !. A trap on ERR, if set, is executed before the shell exits. This option applies to the shell environment and each subshell environment separately \hyperref[sec:commandenvironmentexecution]{see COMMAND EXECUTION ENVIRONMENT above), and may cause subshells to exit before executing all the commands in the subshell.

-f

Disable pathname expansion.

-h

Remember the location of commands as they are looked up for execution. This is enabled by default.

-k

All arguments in the form of assignment statements are placed in the environment for a command, not just those that precede the command name.

-m

Monitor mode. Job control is enabled. This option is on by default for interactive shells on systems that support it \hyperref[sec:jobcontrol]{see JOB CONTROL above). Background processes run in a separate process group and a line containing their exit status is printed upon their completion.

-n

Read commands but do not execute them. This may be used to check a shell script for syntax errors. This is ignored by interactive shells.

-o option-name
The option-name can be one of the following:

allexport

Same as -a.

braceexpand

Same as -B.

emacs

Use an emacs-style command line editing interface. This is enabled by default when the shell is interactive, unless the shell is started with the --noediting option. This also affects the editing interface used for read -e.

errexit

Same as -e.

errtrace

Same as -E.

functrace

Same as -T.

hashall

Same as -h.

histexpand

Same as -H.

history

Enable command history, as described above under HISTORY. This option is on by default in interactive shells.

ignoreeof

The effect is as if the shell command ''IGNOREEOF=10'' had been executed (see Shell Variables above).

keyword

Same as -k.

monitor

Same as -m.

noclobber

Same as -C.

noexec

Same as -n.

noglob

Same as -f.

nolog

Currently ignored.

notify

Same as -b.

nounset

Same as -u.

onecmd

Same as -t.

physical

Same as -P.

pipefail

If set, the return value of a pipeline is the value of the last (rightmost) command to exit with a non-zero status, or zero if all commands in the pipeline exit successfully. This option is disabled by default.

posix

Change the behavior of bash where the default operation differs from the POSIX standard to match the standard (posix mode).

privileged

Same as -p.

verbose

Same as -v.

vi

Use a vi-style command line editing interface. This also affects the editing interface used for read -e.

xtrace

Same as -x.

If -o is supplied with no option-name, the values of the current options are printed. If +o is supplied with no option-name, a series of set commands to recreate the current option settings is displayed on the standard output.

-p
Turn on privileged mode. In this mode, the \$ENV and \$bash\_ENV files are not processed, shell functions are not inherited from the environment, and the SHELLOPTS, BASHOPTS, CDPATH, and GLOBIGNORE variables, if they appear in the environment, are ignored. If the shell is started with the effective user (group) id not equal to the real user (group) id, and the -p option is not supplied, these actions are taken and the effective user id is set to the real user id. If the -p option is supplied at startup, the effective user id is not reset. Turning this option off causes the effective user and group ids to be set to the real user and group ids.

-t

Exit after reading and executing one command.

-u

Treat unset variables and parameters other than the special parameters "@" and "*" as an error when performing parameter expansion. If expansion is attempted on an unset variable or parameter, the shell prints an error message, and, if not interactive, exits with a non-zero status.

-v

Print shell input lines as they are read.

-x

After expanding each simple command, for command, case command, select command, or arithmetic for command, display the expanded value of PS4, followed by the command and its expanded arguments or associated word list.

-B

The shell performs brace expansion (see Brace Expansion above). This is on by default.

-C

If set, bash does not overwrite an existing file with the >, \&, and <> redirection operators. This may be overridden when creating output files by using the redirection operator >| instead of >.

-E

If set, any trap on ERR is inherited by shell functions, command substitutions, and commands executed in a subshell environment. The ERR trap is normally not inherited in such cases.

-H

Enable ! style history substitution. This option is on by default when the shell is interactive.

-P

If set, the shell does not follow symbolic links when executing commands such as cd that change the current working directory. It uses the physical directory structure instead. By default, bash follows the logical chain of directories when performing commands which change the current directory.

-T

If set, any traps on DEBUG and RETURN are inherited by shell functions, command substitutions, and commands executed in a subshell environment. The DEBUG and RETURN traps are normally not inherited in such cases.

--

If no arguments follow this option, then the positional parameters are unset. Otherwise, the positional parameters are set to the args, even if some of them begin with a -.

-

Signal the end of options, cause all remaining args to be assigned to the positional parameters. The -x and -v options are turned off. If there are no args, the positional parameters remain unchanged.

The options are off by default unless otherwise noted. Using + rather than - causes these options to be turned off. The options can also be specified as arguments to an invocation of the shell. The current set of options may be found in \$-. The return status is always true unless an invalid option is encountered.
shift [n]
The positional parameters from n+1 ... are renamed to \$1 .... Parameters represented by the numbers \$\# down to \$\#-n+1 are unset. n must be a non-negative number less than or equal to \$\#. If n is 0, no parameters are changed. If n is not given, it is assumed to be 1. If n is greater than \$\#, the positional parameters are not changed. The return status is greater than zero if n is greater than \$\# or less than zero; otherwise 0.

\url{shopt [-pqsu] [-o] [optname ...]}
Toggle the values of variables controlling optional shell behavior. With no options, or with the -p option, a list of all settable options is displayed, with an indication of whether or not each is set. The -p option causes output to be displayed in a form that may be reused as input. Other options have the following meanings:
\begin{longtable}\defaultlongtable
-s &
Enable (set) each optname. \\

-u &
Disable (unset) each optname. \\

-q &
Suppresses normal output (quiet mode); the return status indicates whether the optname is set or unset. If multiple optname arguments are given with -q, the return status is zero if all optnames are enabled; non-zero otherwise. \\

-o &
Restricts the values of optname to be those defined for the -o option to the set builtin. \\
\end{longtable}

If either -s or -u is used with no optname arguments, the display is limited to those options which are set or unset, respectively. Unless otherwise noted, the shopt options are disabled (unset) by default.
The return status when listing options is zero if all optnames are enabled, non-zero otherwise. When setting or unsetting options, the return status is zero unless an optname is not a valid shell option.

The list of shopt options is:

\begin{longtable}\defaultlongtable

autocd &
If set, a command name that is the name of a directory is executed as if it were the argument to the cd command. This option is only used by interactive shells. \\

cdable\_vars &
If set, an argument to the cd builtin command that is not a directory is assumed to be the name of a variable whose value is the directory to change to. \\

cdspell &
If set, minor errors in the spelling of a directory component in a cd command will be corrected. The errors checked for are transposed characters, a missing character, and one character too many. If a correction is found, the corrected file name is printed, and the command proceeds. This option is only used by interactive shells. \\

checkhash &
If set, bash checks that a command found in the hash table exists before trying to execute it. If a hashed command no longer exists, a normal path search is performed. \\

checkjobs &
If set, bash lists the status of any stopped and running jobs before exiting an interactive shell. If any jobs are running, this causes the exit to be deferred until a second exit is attempted without an intervening command \hyperref[sec:jobcontrol]{(see JOB CONTROL above)}. The shell always postpones exiting if any jobs are stopped. \\

checkwinsize &
If set, bash checks the window size after each command and, if necessary, updates the values of LINES and COLUMNS. \\

cmdhist &
If set, bash attempts to save all lines of a multiple-line command in the same history entry. This allows easy re-editing of multi-line commands. \\

compat31 &
If set, bash changes its behavior to that of version 3.1 with respect to quoted arguments to the conditional command's =~ operator. \\

compat32 &
If set, bash changes its behavior to that of version 3.2 with respect to locale-specific string comparison when using the conditional command's < and > operators. \\

compat40 &
If set, bash changes its behavior to that of version 4.0 with respect to locale-specific string comparison when using the conditional command's < and > operators and the effect of interrupting a command list. \\

dirspell &
If set, bash attempts spelling correction on directory names during word completion if the directory name initially supplied does not exist. \\

dotglob &
If set, bash includes filenames beginning with a '.' in the results of pathname expansion. \\

execfail &
If set, a non-interactive shell will not exit if it cannot execute the file specified as an argument to the exec builtin command. An interactive shell does not exit if exec fails. \\

expand\_aliases &
If set, aliases are expanded as described above under ALIASES. This option is enabled by default for interactive shells. \\

extdebug &
If set, behavior intended for use by debuggers is enabled: \\

\multicolumn{2}{p}{
\begin{enumerate}
\item  The -F option to the declare builtin displays the source file name and line number corresponding to each function name supplied as an argument.
\item  If the command run by the DEBUG trap returns a non-zero value, the next command is skipped and not executed.
\item  If the command run by the DEBUG trap returns a value of 2, and the shell is executing in a subroutine (a shell function or a shell script executed by the . or source builtins), a call to return is simulated.
\item  bash\_ARGC and bash\_ARGV are updated as described in their descriptions above.
\item  Function tracing is enabled: command substitution, shell functions, and subshells invoked with ( command ) inherit the DEBUG and RETURN traps.
\item  Error tracing is enabled: command substitution, shell functions, and subshells invoked with ( command ) inherit the ERROR trap.
\end{enumerate}
}\\

extglob &
If set, the extended pattern matching features described above under Pathname Expansion are enabled. \\

extquote &
If set, \$'string' and \$"string" quoting is performed within \${parameter} expansions enclosed in double quotes. This option is enabled by default. \\

failglob &
If set, patterns which fail to match filenames during pathname expansion result in an expansion error. \\

force\_fignore &
If set, the suffixes specified by the FIGNORE shell variable cause words to be ignored when performing word completion even if the ignored words are the only possible completions. \hyperref[sec:shellvariables]{See SHELL VARIABLES above} for a description of FIGNORE. This option is enabled by default. \\

globstar &
If set, the pattern ** used in a pathname expansion context will match a files and zero or more directories and subdirectories. If the pattern is followed by a /, only directories and subdirectories match. \\

gnu\_errfmt &
If set, shell error messages are written in the standard GNU error message format. \\

histappend &
If set, the history list is appended to the file named by the value of the HISTFILE variable when the shell exits, rather than overwriting the file. \\

histreedit &
If set, and readline is being used, a user is given the opportunity to re-edit a failed history substitution. \\

histverify &
If set, and readline is being used, the results of history substitution are not immediately passed to the shell parser. Instead, the resulting line is loaded into the readline editing buffer, allowing further modification. \\

hostcomplete &
If set, and readline is being used, bash will attempt to perform hostname completion when a word containing a @ is being completed \hyperref[sec:readline]{(see Completing under READLINE above)}. This is enabled by default. \\

huponexit &
If set, bash will send SIGHUP to all jobs when an interactive login shell exits. \\

interactive\_comments &
If set, allow a word beginning with \# to cause that word and all remaining characters on that line to be ignored in an interactive shell \hyperref[sec:comments]{(see COMMENTS above)}. This option is enabled by default. \\

lithist &
If set, and the cmdhist option is enabled, multi-line commands are saved to the history with embedded newlines rather than using semicolon separators where possible. \\

login\_shell &
The shell sets this option if it is started as a login shell \hyperref[sec:invocation]{see INVOCATION above). The value may not be changed. \\

mailwarn &
If set, and a file that bash is checking for mail has been accessed since the last time it was checked, the message ''The mail in mailfile has been read'' is displayed. \\

no\_empty\_cmd\_completion &
If set, and readline is being used, bash will not attempt to search the PATH for possible completions when completion is attempted on an empty line. \\

nocaseglob &
If set, bash matches filenames in a case-insensitive fashion when performing pathname expansion \hyperref[sec:pathnameexpansion]{(see Pathname Expansion above)}. \\

nocasematch &
If set, bash matches patterns in a case-insensitive fashion when performing matching while executing case or [[ conditional commands. \\

nullglob &
If set, bash allows patterns which match no files (see Pathname Expansion above) to expand to a null string, rather than themselves. \\

progcomp &
If set, the programmable completion facilities \hyperref[sec:programmablecompletion]{(see Programmable Completion above)} are enabled. This option is enabled by default. \\

promptvars &
If set, prompt strings undergo parameter expansion, command substitution, arithmetic expansion, and quote removal after being expanded as described in PROMPTING above. This option is enabled by default. &

restricted\_shell &
The shell sets this option if it is started in restricted mode \hyperref[sec:restrictedshell]{see RESTRICTED SHELL below)}. The value may not be changed. This is not reset when the startup files are executed, allowing the startup files to discover whether or not a shell is restricted. \\

shift\_verbose &
If set, the shift builtin prints an error message when the shift count exceeds the number of positional parameters. \\

sourcepath &
If set, the source (.) builtin uses the value of PATH to find the directory containing the file supplied as an argument. This option is enabled by default. \\

xpg\_echo &
If set, the echo builtin expands backslash-escape sequences by default. \\

suspend [-f] &
Suspend the execution of this shell until it receives a SIGCONT signal. When the suspended shell is a background process, it can be restarted by the fg command. For more information, read the JOB CONTROL section. The suspend command can not suspend the login shell. However, when -f option is specified, suspend command can suspend even login shell. The return status is 0 unless the shell is a login shell and -f is not supplied, or if job control is not enabled. \\

test expr \newline [ expr ]
Return a status of 0 or 1 depending on the evaluation of the conditional expression expr. Each operator and operand must be a separate argument. Expressions are composed of the primaries described above under CONDITIONAL EXPRESSIONS. test does not accept any options, nor does it accept and ignore an argument of -- as signifying the end of options.

Expressions may be combined using the following operators, listed in decreasing order of precedence. The evaluation depends on the number of arguments; see below. \\

! expr &
True if expr is false. \\

( expr ) &
Returns the value of expr. This may be used to override the normal precedence of operators. \\

expr1 -a expr2 &
True if both expr1 and expr2 are true. \\

expr1 -o expr2 &
True if either expr1 or expr2 is true. \\

\multicolumn{2}{p}{test and [ evaluate conditional expressions using a set of rules based on the number of arguments.} \\

0 arguments &
The expression is false. \\

\quad1 argument &
The expression is true if and only if the argument is not null. \\

\quad2 arguments &
If the first argument is !, the expression is true if and only if the second argument is null. If the first argument is one of the unary conditional operators listed above under CONDITIONAL EXPRESSIONS, the expression is true if the unary test is true. If the first argument is not a valid unary conditional operator, the expression is false. \\

\quad3 arguments &
If the second argument is one of the binary conditional operators listed above under CONDITIONAL EXPRESSIONS, the result of the expression is the result of the binary test using the first and third arguments as operands. The -a and -o operators are considered binary operators when there are three arguments. If the first argument is !, the value is the negation of the two-argument test using the second and third arguments. If the first argument is exactly ( and the third argument is exactly ), the result is the one-argument test of the second argument. Otherwise, the expression is false. \\

\quad4 arguments &
If the first argument is !, the result is the negation of the three-argument expression composed of the remaining arguments. Otherwise, the expression is parsed and evaluated according to precedence using the rules listed above. \\

\quad5 or more arguments &
The expression is parsed and evaluated according to precedence using the rules listed above. \\

times &
Print the accumulated user and system times for the shell and for processes run from the shell. The return status is 0. \\

trap [-lp] [[arg] sigspec ...] &
The command arg is to be read and executed when the shell receives signal(s) sigspec. If arg is absent (and there is a single sigspec) or -, each specified signal is reset to its original disposition (the value it had upon entrance to the shell). If arg is the null string the signal specified by each sigspec is ignored by the shell and by the commands it invokes. If arg is not present and -p has been supplied, then the trap commands associated with each sigspec are displayed. If no arguments are supplied or if only -p is given, trap prints the list of commands associated with each signal. The -l option causes the shell to print a list of signal names and their corresponding numbers. Each sigspec is either a signal name defined in <signal.h>, or a signal number. Signal names are case insensitive and the SIG prefix is optional.
If a sigspec is EXIT (0) the command arg is executed on exit from the shell. If a sigspec is DEBUG, the command arg is executed before every simple command, for command, case command, select command, every arithmetic for command, and before the first command executes in a shell function \hyperref[sec:shellgrammar]{see SHELL GRAMMAR above). Refer to the description of the extdebug option to the shopt builtin for details of its effect on the DEBUG trap. If a sigspec is RETURN, the command arg is executed each time a shell function or a script executed with the . or source builtins finishes executing.

If a sigspec is ERR, the command arg is executed whenever a simple command has a non-zero exit status, subject to the following conditions. The ERR trap is not executed if the failed command is part of the command list immediately following a while or until keyword, part of the test in an if statement, part of a command executed in a\&\& or || list, or if the command's return value is being inverted via !. These are the same conditions obeyed by the errexit option.

Signals ignored upon entry to the shell cannot be trapped, reset or listed. Trapped signals that are not being ignored are reset to their original values in a subshell or subshell environment when one is created. The return status is false if any sigspec is invalid; otherwise trap returns true. \\

\begin{lstlisting}
type [-aftpP] name [name ...]
\end{lstlisting} &
With no options, indicate how each name would be interpreted if used as a command name. If the -t option is used, type prints a string which is one of alias, keyword, function, builtin, or file if name is an alias, shell reserved word, function, builtin, or disk file, respectively. If the name is not found, then nothing is printed, and an exit status of false is returned. If the -p option is used, type either returns the name of the disk file that would be executed if name were specified as a command name, or nothing if ''type -t name'' would not return file. The -P option forces a PATH search for each name, even if ''type -t name'' would not return file. If a command is hashed, -p and -P print the hashed value, not necessarily the file that appears first in PATH. If the -a option is used, type prints all of the places that contain an executable named name. This includes aliases and functions, if and only if the -p option is not also used. The table of hashed commands is not consulted when using -a. The -f option suppresses shell function lookup, as with the command builtin. type returns true if all of the arguments are found, false if any are not found. \\

\begin{lstlisting}
ulimit [-HSTabcdefilmnpqrstuvx [limit]]
\end{lstlisting} &
Provides control over the resources available to the shell and to processes started by it, on systems that allow such control. The -H and -S options specify that the hard or soft limit is set for the given resource. A hard limit cannot be increased by a non-root user once it is set; a soft limit may be increased up to the value of the hard limit. If neither -H nor -S is specified, both the soft and hard limits are set. The value of limit can be a number in the unit specified for the resource or one of the special values hard, soft, or unlimited, which stand for the current hard limit, the current soft limit, and no limit, respectively. If limit is omitted, the current value of the soft limit of the resource is printed, unless the -H option is given. When more than one resource is specified, the limit name and unit are printed before the value. Other options are interpreted as follows:

\begin{longtable}\defaultlongtable
-a &
All current limits are reported \\

-b &
The maximum socket buffer size \\

-c &
The maximum size of core files created \\

-d &
The maximum size of a process's data segment \\

-e &
The maximum scheduling priority ("nice") \\

-f &
The maximum size of files written by the shell and its children \\

-i &
The maximum number of pending signals \\

-l &
The maximum size that may be locked into memory \\

-m &
The maximum resident set size (many systems do not honor this limit) \\

-n &
The maximum number of open file descriptors (most systems do not allow this value to be set) \\

-p &
The pipe size in 512-byte blocks (this may not be set) \\

-q &
The maximum number of bytes in POSIX message queues \\

-r &
The maximum real-time scheduling priority \\

-s &
The maximum stack size \\

-t &
The maximum amount of cpu time in seconds \\

-u &
The maximum number of processes available to a single user \\

-v &
The maximum amount of virtual memory available to the shell \\

-x &
The maximum number of file locks \\

-T &
The maximum number of threads 

If limit is given, it is the new value of the specified resource (the -a option is display only). If no option is given, then -f is assumed. Values are in 1024-byte increments, except for -t, which is in seconds, -p, which is in units of 512-byte blocks, and -T, -b, -n, and -u, which are unscaled values. The return status is 0 unless an invalid option or argument is supplied, or an error occurs while setting a new limit.\\

umask [-p] [-S] [mode] &
The user file-creation mask is set to mode. If mode begins with a digit, it is interpreted as an octal number; otherwise it is interpreted as a symbolic mode mask similar to that accepted by chmod(1). If mode is omitted, the current value of the mask is printed. The -S option causes the mask to be printed in symbolic form; the default output is an octal number. If the -p option is supplied, and mode is omitted, the output is in a form that may be reused as input. The return status is 0 if the mode was successfully changed or if no mode argument was supplied, and false otherwise. \\

unalias [-a] [name ...] &
Remove each name from the list of defined aliases. If -a is supplied, all alias definitions are removed. The return value is true unless a supplied name is not a defined alias. \\

unset [-fv] [name ...] &
For each name, remove the corresponding variable or function. If no options are supplied, or the -v option is given, each name refers to a shell variable. Read-only variables may not be unset. If -f is specified, each name refers to a shell function, and the function definition is removed. Each unset variable or function is removed from the environment passed to subsequent commands. If any of COMP\_WORDBREAKS, RANDOM, SECONDS, LINENO, HISTCMD, FUNCNAME, GROUPS, or DIRSTACK are unset, they lose their special properties, even if they are subsequently reset. The exit status is true unless a name is readonly. \\

wait [n ...] &
Wait for each specified process and return its termination status. Each n may be a process ID or a job specification; if a job spec is given, all processes in that job's pipeline are waited for. If n is not given, all currently active child processes are waited for, and the return status is zero. If n specifies a non-existent process or job, the return status is 127. Otherwise, the return status is the exit status of the last process or job waited for. \\
\end{longtable}