\section{History}\label{sec:history}
When the -o history option to the set builtin is enabled, the shell provides access to the command history, the list of commands previously typed. The value of the HISTSIZE variable is used as the number of commands to save in a history list. The text of the last HISTSIZE commands (default 500) is saved. The shell stores each command in the history list prior to parameter and variable expansion \hyperref[sec:expansion]{(see EXPANSION above)} but after history expansion is performed, subject to the values of the shell variables HISTIGNORE and HISTCONTROL.

On startup, the history is initialized from the file named by the variable HISTFILE (default ~/.bash\_history). The file named by the value of HISTFILE is truncated, if necessary, to contain no more than the number of lines specified by the value of HISTFILESIZE. When the history file is read, lines beginning with the history comment character followed immediately by a digit are interpreted as timestamps for the preceding history line. These timestamps are optionally displayed depending on the value of the HISTTIMEFORMAT variable. When an interactive shell exits, the last \$HISTSIZE lines are copied from the history list to \$HISTFILE. If the histappend shell option is enabled \hyperref[sec:shellbuiltincommands]{(see the description of shopt under SHELL BUILTIN COMMANDS below)}, the lines are appended to the history file, otherwise the history file is overwritten. If HISTFILE is unset, or if the history file is unwritable, the history is not saved. If the HISTTIMEFORMAT variable is set, time stamps are written to the history file, marked with the history comment character, so they may be preserved across shell sessions. This uses the history comment character to distinguish timestamps from other history lines. After saving the history, the history file is truncated to contain no more than HISTFILESIZE lines. If HISTFILESIZE is not set, no truncation is performed.

The builtin command fc \hyperref[sec:shellbuiltincommands]{(see SHELL BUILTIN COMMANDS below)} may be used to list or edit and re-execute a portion of the history list. The history builtin may be used to display or modify the history list and manipulate the history file. When using command-line editing, search commands are available in each editing mode that provide access to the history list.

The shell allows control over which commands are saved on the history list. The HISTCONTROL and HISTIGNORE variables may be set to cause the shell to save only a subset of the commands entered. The cmdhist shell option, if enabled, causes the shell to attempt to save each line of a multi-line command in the same history entry, adding semicolons where necessary to preserve syntactic correctness. The lithist shell option causes the shell to save the command with embedded newlines instead of semicolons. See the description of the shopt builtin below under \hyperref[sec:shellbuiltincommands]{SHELL BUILTIN COMMANDS} for information on setting and unsetting shell options.

\subsection{History Expansion}\label{sec:historyexpansion}
The shell supports a history expansion feature that is similar to the history expansion in csh. This section describes what syntax features are available. This feature is enabled by default for interactive shells, and can be disabled using the +H option to the set builtin command \hyperref[sec:shellbuiltincommands]{(see SHELL BUILTIN COMMANDS below)}. Non-interactive shells do not perform history expansion by default.

History expansions introduce words from the history list into the input stream, making it easy to repeat commands, insert the arguments to a previous command into the current input line, or fix errors in previous commands quickly.

History expansion is performed immediately after a complete line is read, before the shell breaks it into words. It takes place in two parts. The first is to determine which line from the history list to use during substitution. The second is to select portions of that line for inclusion into the current one. The line selected from the history is the event, and the portions of that line that are acted upon are words. Various modifiers are available to manipulate the selected words. The line is broken into words in the same fashion as when reading input, so that several metacharacter-separated words surrounded by quotes are considered one word. History expansions are introduced by the appearance of the history expansion character, which is ! by default. Only backslash (\) and single quotes can quote the history expansion character.

Several characters inhibit history expansion if found immediately following the history expansion character, even if it is unquoted: space, tab, newline, carriage return, and =. If the extglob shell option is enabled, ( will also inhibit expansion.

Several shell options settable with the shopt builtin may be used to tailor the behavior of history expansion. If the histverify shell option is enabled (see the description of the shopt builtin below), and readline is being used, history substitutions are not immediately passed to the shell parser. Instead, the expanded line is reloaded into the readline editing buffer for further modification. If readline is being used, and the histreedit shell option is enabled, a failed history substitution will be reloaded into the readline editing buffer for correction. The -p option to the history builtin command may be used to see what a history expansion will do before using it. The -s option to the history builtin may be used to add commands to the end of the history list without actually executing them, so that they are available for subsequent recall.

The shell allows control of the various characters used by the history expansion mechanism (see the description of histchars above under Shell Variables). The shell uses the history comment character to mark history timestamps when writing the history file.