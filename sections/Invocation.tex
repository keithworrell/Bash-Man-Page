\section{Invocation}\label{sec:invocation}
A login shell is one whose first character of argument zero is a -, or one started with the --login option.

An interactive shell is one started without non-option arguments and without the \textbf{-c} option whose standard input and error are both connected to terminals (as determined by \gls{isatty}), or one started with the \textbf{-i} option. \gls{PS1} is set and \textbf{\$-} includes \textbf{i} if \gls{bash} is interactive, allowing a shell script or a startup file to test this state.

The following paragraphs describe how \gls{bash} executes its startup files. If any of the files exist but cannot be read, \gls{bash} reports an error. Tildes are expanded in file names as described below under Tilde Expansion in the \hyperref[sec:expansion]{\textbf{EXPANSION} section}.

When \gls{bash} is invoked as an interactive login shell, or as a non-interactive shell with the \textbf{--login} option, it first reads and executes commands from the file /etc/profile, if that file exists. After reading that file, it looks for \url{~/.bash_profile}, \url{~/.bash_login}, and \url{~/.profile}, in that order, and reads and executes commands from the first one that exists and is readable. The --noprofile option may be used when the shell is started to inhibit this behavior.

When a login shell exits, \gls{bash} reads and executes commands from the files \url{~/.bash_logout} and \url{/etc/bash.bash_logout}, if the files exists.

When an interactive shell that is not a login shell is started, \gls{bash} reads and executes commands from \url{~/.bashrc}, if that file exists. This may be inhibited by using the --norc option. The \url{--rcfile} file option will force \gls{bash} to read and execute commands from file instead of \url{~/.bashrc}.

When \gls{bash} is started non-interactively, to run a shell script, for example, it looks for the variable \url{BASH_ENV} in the environment, expands its value if it appears there, and uses the expanded value as the name of a file to read and execute.  \Gls{bash} behaves as if the following command were executed:

\begin{lstlisting}[language=bash]
    if [ -n "\$bash\_ENV" ]; then . "\$bash\_ENV"; fi
\end{lstlisting}
but the value of the  \gls{PATH} variable is not used to search for the file name.

If \gls{bash} is invoked with the name sh, it tries to mimic the startup behavior of historical versions of sh as closely as possible, while conforming to the  \gls{POSIX} standard as well. When invoked as an interactive login shell, or a non-interactive shell with the \url{--login} option, it first attempts to read and execute commands from /etc/profile and \url{~/.profile}, in that order. The \url{--noprofile} option may be used to inhibit this behavior. When invoked as an interactive shell with the name sh, \gls{bash} looks for the variable ENV, expands its value if it is defined, and uses the expanded value as the name of a file to read and execute. Since a shell invoked as sh does not attempt to read and execute commands from any other startup files, the \url{--rcfile} option has no effect. A non-interactive shell invoked with the name sh does not attempt to read any other startup files. When invoked as sh, \gls{bash} enters posix mode after the startup files are read.

When \gls{bash} is started in posix mode, as with the --posix command line option, it follows the  \gls{POSIX} standard for startup files. In this mode, interactive shells expand the \gls{ENV} variable and commands are read and executed from the file whose name is the expanded value. No other startup files are read.

\Gls{bash} attempts to determine when it is being run with its standard input connected to a a network connection, as if by the remote shell daemon, usually \textit{rshd}, or the secure shell daemon \textit{sshd}. If \gls{bash} determines it is being run in this fashion, it reads and executes commands from \url{~/.bashrc}, if that file exists and is readable. It will not do this if invoked as sh. The \textbf{--norc} option may be used to inhibit this behavior, and the \url{--rcfile} option may be used to force another file to be read, but \textbf{rshd} does not generally invoke the shell with those options or allow them to be specified.

If the shell is started with the effective user (group) id not equal to the real user (group) id, and the \textbf{-p} option is not supplied, no startup files are read, shell functions are not inherited from the environment, the \textbf{SHELLOPTS}, \textbf{BASHOPTS}, \textbf{CDPATH}, and \textbf{GLOBIGNORE} variables, if they appear in the environment, are ignored, and the effective user id is set to the real user id. If the \textbf{-p} option is supplied at invocation, the startup behavior is the same, but the effective user id is not reset.