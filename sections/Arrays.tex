\section{Arrays}\label{sec:arrays}

Bash provides one-dimensional indexed and associative array variables. Any variable may be used as an indexed array; the declare builtin will explicitly declare an array. There is no maximum limit on the size of an array, nor any requirement that members be indexed or assigned contiguously. Indexed arrays are referenced using integers (including arithmetic expressions) and are zero-based; associative arrays are referenced using arbitrary strings.
An indexed array is created automatically if any variable is assigned to using the syntax name[subscript]=value. The subscript is treated as an arithmetic expression that must evaluate to a number greater than or equal to zero. To explicitly declare an indexed array, use declare -a name \hyperref[sec:shellbuiltincommands]{see SHELL BUILTIN COMMANDS below)}. declare -a name[subscript] is also accepted; the subscript is ignored.

Associative arrays are created using declare -A name.

Attributes may be specified for an array variable using the declare and readonly builtins. Each attribute applies to all members of an array.

Arrays are assigned to using compound assignments of the form name=(value1 ... valuen), where each value is of the form [subscript]=string. Indexed array assignments do not require the bracket and subscript. When assigning to indexed arrays, if the optional brackets and subscript are supplied, that index is assigned to; otherwise the index of the element assigned is the last index assigned to by the statement plus one. Indexing starts at zero.

When assigning to an associative array, the subscript is required.

This syntax is also accepted by the declare builtin. Individual array elements may be assigned to using the name[subscript]=value syntax introduced above.

Any element of an array may be referenced using \${name[subscript]}. The braces are required to avoid conflicts with pathname expansion. If subscript is @ or *, the word expands to all members of name. These subscripts differ only when the word appears within double quotes. If the word is double-quoted, \${name[*]} expands to a single word with the value of each array member separated by the first character of the IFS special variable, and \${name[@]} expands each element of name to a separate word. When there are no array members, \${name[@]} expands to nothing. If the double-quoted expansion occurs within a word, the expansion of the first parameter is joined with the beginning part of the original word, and the expansion of the last parameter is joined with the last part of the original word. This is analogous to the expansion of the special parameters * and @ (see Special Parameters above). \${\#name[subscript]} expands to the length of \${name[subscript]}. If subscript is * or @, the expansion is the number of elements in the array. Referencing an array variable without a subscript is equivalent to referencing the array with a subscript of 0.

An array variable is considered set if a subscript has been assigned a value. The null string is a valid value.

The unset builtin is used to destroy arrays. unset name[subscript] destroys the array element at index subscript. Care must be taken to avoid unwanted side effects caused by pathname expansion. unset name, where name is an array, or unset name[subscript], where subscript is * or @, removes the entire array.

The declare, local, and readonly builtins each accept a -a option to specify an indexed array and a -A option to specify an associative array. The read builtin accepts a -a option to assign a list of words read from the standard input to an array. The set and declare builtins display array values in a way that allows them to be reused as assignments.