\section{Functions}\label{sec:functions}
A shell function, defined as described above under SHELL GRAMMAR, stores a series of commands for later execution. When the name of a shell function is used as a simple command name, the list of commands associated with that function name is executed. Functions are executed in the context of the current shell; no new process is created to interpret them (contrast this with the execution of a shell script). When a function is executed, the arguments to the function become the positional parameters during its execution. The special parameter \# is updated to reflect the change. Special parameter 0 is unchanged. The first element of the FUNCNAME variable is set to the name of the function while the function is executing.

All other aspects of the shell execution environment are identical between a function and its caller with these exceptions: the DEBUG and RETURN traps (see the description of the trap builtin under SHELL BUILTIN COMMANDS below)} are not inherited unless the function has been given the trace attribute (see the description of the declare builtin below) or the -o functrace shell option has been enabled with the set builtin (in which case all functions inherit the DEBUG and RETURN traps), and the ERR trap is not inherited unless the -o errtrace shell option has been enabled.

Variables local to the function may be declared with the local builtin command. Ordinarily, variables and their values are shared between the function and its caller.

If the builtin command return is executed in a function, the function completes and execution resumes with the next command after the function call. Any command associated with the RETURN trap is executed before execution resumes. When a function completes, the values of the positional parameters and the special parameter \# are restored to the values they had prior to the function's execution.

Function names and definitions may be listed with the -f option to the declare or typeset builtin commands. The -F option to declare or typeset will list the function names only (and optionally the source file and line number, if the extdebug shell option is enabled). Functions may be exported so that subshells automatically have them defined with the -f option to the export builtin. A function definition may be deleted using the -f option to the unset builtin. Note that shell functions and variables with the same name may result in multiple identically-named entries in the environment passed to the shell's children. Care should be taken in cases where this may cause a problem.

Functions may be recursive. No limit is imposed on the number of recursive calls.