\section{Event Designators}\label{sec:eventdesignators}

An event designator is a reference to a command line entry in the history list.

\begin{longtable}\defaultlongtable
! &
Start a history substitution, except when followed by a blank, newline, carriage return, = or ( (when the extglob shell option is enabled using the shopt builtin). \\

!n &
Refer to command line n. \\

!-n &
Refer to the current command line minus n. \\

!! &
Refer to the previous command. This is a synonym for '\!-1'. \\

!string &
Refer to the most recent command starting with string. \\

!?string[?] &
Refer to the most recent command containing string. The trailing ? may be omitted if string is followed immediately by a newline. \\

\^{} string1 \^{} string2 \^{} &
Quick substitution. Repeat the last command, replacing string1 with string2. Equivalent to ''!!:s/string1/string2/'' \hyperref[sec:modifiers]{(see Modifiers below)}. \\

\!\# &
The entire command line typed so far. \\
\end{longtable}

\subsection{Word Designators}\label{sec:worddesignators}

    Word designators are used to select desired words from the event. A : separates the event specification from the word designator. It may be omitted if the word designator begins with a \^{}, \$, *, -, or \%. Words are numbered from the beginning of the line, with the first word being denoted by 0 (zero). Words are inserted into the current line separated by single spaces.
    
    \begin{longtable}\defaultlongtable
        0 (zero) &
        The zeroth word. For the shell, this is the command word.\\
        
        n &
        The nth word. \\
        
        \^{} &
        The first argument. That is, word 1. \\
        
        \$ &
        The last argument. \\
        
        \% &
        The word matched by the most recent '?string?' search. \\
        
        x-y &
        A range of words; '-y' abbreviates '0-y'. \\
        
        * &
        All of the words but the zeroth. This is a synonym for '1-\$'. It is not an error to use * if there is just one word in the event; the empty string is returned in that case. \\
        
        x* &
        Abbreviates x-\$. \\
        
        x-
        Abbreviates x-\$ like x*, but omits the last word. \\
    \end{longtable}
    
    If a word designator is supplied without an event specification, the previous command is used as the event.
    
\subsection{Modifiers}\label{sec:modifiers}
    After the optional word designator, there may appear a sequence of one or more of the following modifiers, each preceded by a ':'.
    
    \begin{longtable}\defaultlongtable
        h &
        Remove a trailing file name component, leaving only the head. \\
        
        t &
        Remove all leading file name components, leaving the tail. \\
        
        r &
        Remove a trailing suffix of the form .xxx, leaving the basename. \\
        
        e &
        Remove all but the trailing suffix. \\
        
        p &
        Print the new command but do not execute it. \\
        
        q &
        Quote the substituted words, escaping further substitutions. \\
        
        x &
        Quote the substituted words as with q, but break into words at blanks and newlines. \\
        
        s/old/new/ &
        Substitute new for the first occurrence of old in the event line. Any delimiter can be used in place of /. The final delimiter is optional if it is the last character of the event line. The delimiter may be quoted in old and new with a single backslash. If\& appears in new, it is replaced by old. A single backslash will quote the\&. If old is null, it is set to the last old substituted, or, if no previous history substitutions took place, the last string in a !?string[?] search. \\
        
        \& &
        Repeat the previous substitution. \\
        
        g &
        Cause changes to be applied over the entire event line. This is used in conjunction with ':s' (e.g., ':gs/old/new/') or '\&'. If used with ':s', any delimiter can be used in place of /, and the final delimiter is optional if it is the last character of the event line. An a may be used as a synonym for g. \\
        
        G &
        Apply the following 's' modifier once to each word in the event line. \\
    \end{longtable}
