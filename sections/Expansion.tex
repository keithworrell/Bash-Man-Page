\section{Expansion}\label{sec:expansion}
Expansion is performed on the command line after it has been split into words. There are seven kinds of expansion performed: brace expansion, tilde expansion, parameter and variable expansion, command substitution, arithmetic expansion, word splitting, and pathname expansion.

The order of expansions is: brace expansion, tilde expansion, parameter, variable and arithmetic expansion and command substitution (done in a left-to-right fashion), word splitting, and pathname expansion.

On systems that can support it, there is an additional expansion available: process substitution.

Only brace expansion, word splitting, and pathname expansion can change the number of words of the expansion; other expansions expand a single word to a single word. The only exceptions to this are the expansions of "\$@" and "\${name[@]}" as explained above (see PARAMETERS).

Brace Expansion

Brace expansion is a mechanism by which arbitrary strings may be generated. This mechanism is similar to pathname expansion, but the filenames generated need not exist. Patterns to be brace expanded take the form of an optional preamble, followed by either a series of comma-separated strings or a sequence expression between a pair of braces, followed by an optional postscript. The preamble is prefixed to each string contained within the braces, and the postscript is then appended to each resulting string, expanding left to right.
Brace expansions may be nested. The results of each expanded string are not sorted; left to right order is preserved. For example, a{d,c,b}e expands into 'ade ace abe'.

A sequence expression takes the form {x..y[..incr]}, where x and y are either integers or single characters, and incr, an optional increment, is an integer. When integers are supplied, the expression expands to each number between x and y, inclusive. Supplied integers may be prefixed with 0 to force each term to have the same width. When either x or y begins with a zero, the shell attempts to force all generated terms to contain the same number of digits, zero-padding where necessary. When characters are supplied, the expression expands to each character lexicographically between x and y, inclusive. Note that both x and y must be of the same type. When the increment is supplied, it is used as the difference between each term. The default increment is 1 or -1 as appropriate.

Brace expansion is performed before any other expansions, and any characters special to other expansions are preserved in the result. It is strictly textual. Bash does not apply any syntactic interpretation to the context of the expansion or the text between the braces.

A correctly-formed brace expansion must contain unquoted opening and closing braces, and at least one unquoted comma or a valid sequence expression. Any incorrectly formed brace expansion is left unchanged. A { or , may be quoted with a backslash to prevent its being considered part of a brace expression. To avoid conflicts with parameter expansion, the string \${ is not considered eligible for brace expansion.

This construct is typically used as shorthand when the common prefix of the strings to be generated is longer than in the above example:

\begin{lstlisting}
mkdir /usr/local/src/bash/{old,new,dist,bugs}
\end{lstlisting}

or

\begin{lstlisting}
chown root /usr/{ucb/{ex,edit},lib/{ex?.?*,how\_ex}}
\end{lstlisting}

Brace expansion introduces a slight incompatibility with historical versions of sh. sh does not treat opening or closing braces specially when they appear as part of a word, and preserves them in the output. Bash removes braces from words as a consequence of brace expansion. For example, a word entered to sh as file{1,2} appears identically in the output. The same word is output as file1 file2 after expansion by bash. If strict compatibility with sh is desired, start bash with the +B option or disable brace expansion with the +B option to the set command \hyperref[sec:shellbuiltincommands]{see SHELL BUILTIN COMMANDS below)}.
Tilde Expansion

If a word begins with an unquoted tilde character ('~'), all of the characters preceding the first unquoted slash (or all characters, if there is no unquoted slash) are considered a tilde-prefix. If none of the characters in the tilde-prefix are quoted, the characters in the tilde-prefix following the tilde are treated as a possible login name. If this login name is the null string, the tilde is replaced with the value of the shell parameter HOME. If HOME is unset, the home directory of the user executing the shell is substituted instead. Otherwise, the tilde-prefix is replaced with the home directory associated with the specified login name.
If the tilde-prefix is a '~+', the value of the shell variable PWD replaces the tilde-prefix. If the tilde-prefix is a '~-', the value of the shell variable OLDPWD, if it is set, is substituted. If the characters following the tilde in the tilde-prefix consist of a number N, optionally prefixed by a '+' or a '-', the tilde-prefix is replaced with the corresponding element from the directory stack, as it would be displayed by the dirs builtin invoked with the tilde-prefix as an argument. If the characters following the tilde in the tilde-prefix consist of a number without a leading '+' or '-', '+' is assumed.

If the login name is invalid, or the tilde expansion fails, the word is unchanged.

Each variable assignment is checked for unquoted tilde-prefixes immediately following a : or the first =. In these cases, tilde expansion is also performed. Consequently, one may use file names with tildes in assignments to PATH, MAILPATH, and CDPATH, and the shell assigns the expanded value.

Parameter Expansion

The '\$' character introduces parameter expansion, command substitution, or arithmetic expansion. The parameter name or symbol to be expanded may be enclosed in braces, which are optional but serve to protect the variable to be expanded from characters immediately following it which could be interpreted as part of the name.
When braces are used, the matching ending brace is the first '}' not escaped by a backslash or within a quoted string, and not within an embedded arithmetic expansion, command substitution, or parameter expansion.

\${parameter}
The value of parameter is substituted. The braces are required when parameter is a positional parameter with more than one digit, or when parameter is followed by a character which is not to be interpreted as part of its name.
If the first character of parameter is an exclamation point (!), a level of variable indirection is introduced. Bash uses the value of the variable formed from the rest of parameter as the name of the variable; this variable is then expanded and that value is used in the rest of the substitution, rather than the value of parameter itself. This is known as indirect expansion. The exceptions to this are the expansions of \${!prefix*} and \${!name[@]} described below. The exclamation point must immediately follow the left brace in order to introduce indirection.
In each of the cases below, word is subject to tilde expansion, parameter expansion, command substitution, and arithmetic expansion.

When not performing substring expansion, using the forms documented below, bash tests for a parameter that is unset or null. Omitting the colon results in a test only for a parameter that is unset.

\${parameter:-word}
Use Default Values. If parameter is unset or null, the expansion of word is substituted. Otherwise, the value of parameter is substituted.
\${parameter:=word}
Assign Default Values. If parameter is unset or null, the expansion of word is assigned to parameter. The value of parameter is then substituted. Positional parameters and special parameters may not be assigned to in this way.
\${parameter:?word}
Display Error if Null or Unset. If parameter is null or unset, the expansion of word (or a message to that effect if word is not present) is written to the standard error and the shell, if it is not interactive, exits. Otherwise, the value of parameter is substituted.
\${parameter:+word}
Use Alternate Value. If parameter is null or unset, nothing is substituted, otherwise the expansion of word is substituted.
\${parameter:offset}
\${parameter:offset:length}
Substring Expansion. Expands to up to length characters of parameter starting at the character specified by offset. If length is omitted, expands to the substring of parameter starting at the character specified by offset. length and offset are arithmetic expressions \hyperref[sec:arithmeticevaluation]{see ARITHMETIC EVALUATION below)}. length must evaluate to a number greater than or equal to zero. If offset evaluates to a number less than zero, the value is used as an offset from the end of the value of parameter. If parameter is @, the result is length positional parameters beginning at offset. If parameter is an indexed array name subscripted by @ or *, the result is the length members of the array beginning with \${parameter[offset]}. A negative offset is taken relative to one greater than the maximum index of the specified array. Substring expansion applied to an associative array produces undefined results. Note that a negative offset must be separated from the colon by at least one space to avoid being confused with the :- expansion. Substring indexing is zero-based unless the positional parameters are used, in which case the indexing starts at 1 by default. If offset is 0, and the positional parameters are used, \$0 is prefixed to the list.
\${!prefix*}
\${!prefix@}
Names matching prefix. Expands to the names of variables whose names begin with prefix, separated by the first character of the IFS special variable. When @ is used and the expansion appears within double quotes, each variable name expands to a separate word.
\${!name[@]}
\${!name[*]}
List of array keys. If name is an array variable, expands to the list of array indices (keys) assigned in name. If name is not an array, expands to 0 if name is set and null otherwise. When @ is used and the expansion appears within double quotes, each key expands to a separate word.
\${\#parameter}
Parameter length. The length in characters of the value of parameter is substituted. If parameter is * or @, the value substituted is the number of positional parameters. If parameter is an array name subscripted by * or @, the value substituted is the number of elements in the array.
\${parameter\#word}
\${parameter\#\#word}
Remove matching prefix pattern. The word is expanded to produce a pattern just as in pathname expansion. If the pattern matches the beginning of the value of parameter, then the result of the expansion is the expanded value of parameter with the shortest matching pattern (the ''\#'' case) or the longest matching pattern (the ''\#\#'' case) deleted. If parameter is @ or *, the pattern removal operation is applied to each positional parameter in turn, and the expansion is the resultant list. If parameter is an array variable subscripted with @ or *, the pattern removal operation is applied to each member of the array in turn, and the expansion is the resultant list.
\${parameter%word}
\${parameter%%word}
Remove matching suffix pattern. The word is expanded to produce a pattern just as in pathname expansion. If the pattern matches a trailing portion of the expanded value of parameter, then the result of the expansion is the expanded value of parameter with the shortest matching pattern (the ''%'' case) or the longest matching pattern (the ''%%'' case) deleted. If parameter is @ or *, the pattern removal operation is applied to each positional parameter in turn, and the expansion is the resultant list. If parameter is an array variable subscripted with @ or *, the pattern removal operation is applied to each member of the array in turn, and the expansion is the resultant list.
\${parameter/pattern/string}
Pattern substitution. The pattern is expanded to produce a pattern just as in pathname expansion. Parameter is expanded and the longest match of pattern against its value is replaced with string. If pattern begins with /, all matches of pattern are replaced with string. Normally only the first match is replaced. If pattern begins with \#, it must match at the beginning of the expanded value of parameter. If pattern begins with \%, it must match at the end of the expanded value of parameter. If string is null, matches of pattern are deleted and the / following pattern may be omitted. If parameter is @ or *, the substitution operation is applied to each positional parameter in turn, and the expansion is the resultant list. If parameter is an array variable subscripted with @ or *, the substitution operation is applied to each member of the array in turn, and the expansion is the resultant list.
\${parameter\^{}pattern}
\${parameter\^{}\^{}pattern}
\${parameter,pattern}
\${parameter,,pattern}
Case modification. This expansion modifies the case of alphabetic characters in parameter. The pattern is expanded to produce a pattern just as in pathname expansion. The \^{} operator converts lowercase letters matching pattern to uppercase; the , operator converts matching uppercase letters to lowercase. The \^{}\^{} and ,, expansions convert each matched character in the expanded value; the \^{} and , expansions match and convert only the first character in the expanded value.. If pattern is omitted, it is treated like a ?, which matches every character. If parameter is @ or *, the case modification operation is applied to each positional parameter in turn, and the expansion is the resultant list. If parameter is an array variable subscripted with @ or *, the case modification operation is applied to each member of the array in turn, and the expansion is the resultant list.
Command Substitution

Command substitution allows the output of a command to replace the command name. There are two forms:
\$(command)
or
`command`
Bash performs the expansion by executing command and replacing the command substitution with the standard output of the command, with any trailing newlines deleted. Embedded newlines are not deleted, but they may be removed during word splitting. The command substitution \$(cat file) can be replaced by the equivalent but faster \$(< file).
When the old-style backquote form of substitution is used, backslash retains its literal meaning except when followed by \$, `, or \. The first backquote not preceded by a backslash terminates the command substitution. When using the \$(command) form, all characters between the parentheses make up the command; none are treated specially.

Command substitutions may be nested. To nest when using the backquoted form, escape the inner backquotes with backslashes.

If the substitution appears within double quotes, word splitting and pathname expansion are not performed on the results.

Arithmetic Expansion

Arithmetic expansion allows the evaluation of an arithmetic expression and the substitution of the result. The format for arithmetic expansion is:
\$((expression))
The expression is treated as if it were within double quotes, but a double quote inside the parentheses is not treated specially. All tokens in the expression undergo parameter expansion, string expansion, command substitution, and quote removal. Arithmetic expansions may be nested.
The evaluation is performed according to the rules listed below under ARITHMETIC EVALUATION. If expression is invalid, bash prints a message indicating failure and no substitution occurs.

Process Substitution

Process substitution is supported on systems that support named pipes (FIFOs) or the /dev/fd method of naming open files. It takes the form of <(list) or >(list). The process list is run with its input or output connected to a FIFO or some file in /dev/fd. The name of this file is passed as an argument to the current command as the result of the expansion. If the >(list) form is used, writing to the file will provide input for list. If the <(list) form is used, the file passed as an argument should be read to obtain the output of list.
When available, process substitution is performed simultaneously with parameter and variable expansion, command substitution, and arithmetic expansion.

Word Splitting

The shell scans the results of parameter expansion, command substitution, and arithmetic expansion that did not occur within double quotes for word splitting.
The shell treats each character of IFS as a delimiter, and splits the results of the other expansions into words on these characters. If IFS is unset, or its value is exactly <space><tab><newline>, the default, then sequences of <space>, <tab>, and <newline> at the beginning and end of the results of the previous expansions are ignored, and any sequence of IFS characters not at the beginning or end serves to delimit words. If IFS has a value other than the default, then sequences of the whitespace characters space and tab are ignored at the beginning and end of the word, as long as the whitespace character is in the value of IFS (an IFS whitespace character). Any character in IFS that is not IFS whitespace, along with any adjacent IFS whitespace characters, delimits a field. A sequence of IFS whitespace characters is also treated as a delimiter. If the value of IFS is null, no word splitting occurs.

Explicit null arguments ("" or '') are retained. Unquoted implicit null arguments, resulting from the expansion of parameters that have no values, are removed. If a parameter with no value is expanded within double quotes, a null argument results and is retained.

Note that if no expansion occurs, no splitting is performed.

Pathname Expansion

After word splitting, unless the -f option has been set, bash scans each word for the characters *, ?, and [. If one of these characters appears, then the word is regarded as a pattern, and replaced with an alphabetically sorted list of file names matching the pattern. If no matching file names are found, and the shell option nullglob is not enabled, the word is left unchanged. If the nullglob option is set, and no matches are found, the word is removed. If the failglob shell option is set, and no matches are found, an error message is printed and the command is not executed. If the shell option nocaseglob is enabled, the match is performed without regard to the case of alphabetic characters. When a pattern is used for pathname expansion, the character ''.'' at the start of a name or immediately following a slash must be matched explicitly, unless the shell option dotglob is set. When matching a pathname, the slash character must always be matched explicitly. In other cases, the ''.'' character is not treated specially. See the description of shopt below under SHELL BUILTIN COMMANDS for a description of the nocaseglob, nullglob, failglob, and dotglob shell options.
The GLOBIGNORE shell variable may be used to restrict the set of file names matching a pattern. If GLOBIGNORE is set, each matching file name that also matches one of the patterns in GLOBIGNORE is removed from the list of matches. The file names ''.'' and ''..'' are always ignored when GLOBIGNORE is set and not null. However, setting GLOBIGNORE to a non-null value has the effect of enabling the dotglob shell option, so all other file names beginning with a ''.'' will match. To get the old behavior of ignoring file names beginning with a ''.'', make ''.*'' one of the patterns in GLOBIGNORE. The dotglob option is disabled when GLOBIGNORE is unset.

Pattern Matching

Any character that appears in a pattern, other than the special pattern characters described below, matches itself. The NUL character may not occur in a pattern. A backslash escapes the following character; the escaping backslash is discarded when matching. The special pattern characters must be quoted if they are to be matched literally.

The special pattern characters have the following meanings:

*
Matches any string, including the null string. When the globstar shell option is enabled, and * is used in a pathname expansion context, two adjacent *s used as a single pattern will match all files and zero or more directories and subdirectories. If followed by a /, two adjacent *s will match only directories and subdirectories.

?

Matches any single character.

[...]

Matches any one of the enclosed characters. A pair of characters separated by a hyphen denotes a range expression; any character that sorts between those two characters, inclusive, using the current locale's collating sequence and character set, is matched. If the first character following the [ is a ! or a \^{} then any character not enclosed is matched. The sorting order of characters in range expressions is determined by the current locale and the value of the LC\_COLLATE shell variable, if set. A - may be matched by including it as the first or last character in the set. A ] may be matched by including it as the first character in the set.

Within [ and ], character classes can be specified using the syntax [:class:], where class is one of the following classes defined in the POSIX standard:
alnum alpha ascii blank cntrl digit graph lower print punct space upper word xdigit
A character class matches any character belonging to that class. The word character class matches letters, digits, and the character \_.
Within [ and ], an equivalence class can be specified using the syntax [=c=], which matches all characters with the same collation weight (as defined by the current locale) as the character c.

Within [ and ], the syntax [.symbol.] matches the collating symbol symbol.

If the extglob shell option is enabled using the shopt builtin, several extended pattern matching operators are recognized. In the following description, a pattern-list is a list of one or more patterns separated by a |. Composite patterns may be formed using one or more of the following sub-patterns:
?(pattern-list)
Matches zero or one occurrence of the given patterns

*(pattern-list)
Matches zero or more occurrences of the given patterns

+(pattern-list)
Matches one or more occurrences of the given patterns

@(pattern-list)
Matches one of the given patterns

!(pattern-list)
Matches anything except one of the given patterns

Quote Removal

After the preceding expansions, all unquoted occurrences of the characters \, ', and " that did not result from one of the above expansions are removed.